\documentclass[11pt,a4paper]{article}
\usepackage{lmodern}

\usepackage{amssymb,amsmath}
\usepackage{ifxetex,ifluatex}
\usepackage{fixltx2e} % provides \textsubscript
\ifnum 0\ifxetex 1\fi\ifluatex 1\fi=0 % if pdftex
  \usepackage[T1]{fontenc}
  \usepackage[utf8]{inputenc}
\else % if luatex or xelatex
  \ifxetex
    \usepackage{mathspec}
    \usepackage{xltxtra,xunicode}
  \else
    \usepackage{fontspec}
  \fi
  \defaultfontfeatures{Mapping=tex-text,Scale=MatchLowercase}
  \newcommand{\euro}{€}
\fi
% use upquote if available, for straight quotes in verbatim environments
\IfFileExists{upquote.sty}{\usepackage{upquote}}{}
% use microtype if available
\IfFileExists{microtype.sty}{%
\usepackage{microtype}
\UseMicrotypeSet[protrusion]{basicmath} % disable protrusion for tt fonts
}{}
\usepackage[lmargin=2.5cm,rmargin=2.5cm,tmargin=2.5cm,bmargin=2.5cm]{geometry}

% Figure Placement:
\usepackage{float}
\let\origfigure\figure
\let\endorigfigure\endfigure
\renewenvironment{figure}[1][2] {
    \expandafter\origfigure\expandafter[H]
} {
    \endorigfigure
}

%% citation setup

\usepackage{csquotes}

\usepackage[backend=biber, maxbibnames = 99, style = apa]{biblatex}
\setlength\bibitemsep{1.5\itemsep}
\bibliography{references.bib}
\usepackage{color}
\usepackage{fancyvrb}
\newcommand{\VerbBar}{|}
\newcommand{\VERB}{\Verb[commandchars=\\\{\}]}
\DefineVerbatimEnvironment{Highlighting}{Verbatim}{commandchars=\\\{\}}
% Add ',fontsize=\small' for more characters per line
\usepackage{framed}
\definecolor{shadecolor}{RGB}{248,248,248}
\newenvironment{Shaded}{\begin{snugshade}}{\end{snugshade}}
\newcommand{\AlertTok}[1]{\textcolor[rgb]{0.94,0.16,0.16}{#1}}
\newcommand{\AnnotationTok}[1]{\textcolor[rgb]{0.56,0.35,0.01}{\textbf{\textit{#1}}}}
\newcommand{\AttributeTok}[1]{\textcolor[rgb]{0.77,0.63,0.00}{#1}}
\newcommand{\BaseNTok}[1]{\textcolor[rgb]{0.00,0.00,0.81}{#1}}
\newcommand{\BuiltInTok}[1]{#1}
\newcommand{\CharTok}[1]{\textcolor[rgb]{0.31,0.60,0.02}{#1}}
\newcommand{\CommentTok}[1]{\textcolor[rgb]{0.56,0.35,0.01}{\textit{#1}}}
\newcommand{\CommentVarTok}[1]{\textcolor[rgb]{0.56,0.35,0.01}{\textbf{\textit{#1}}}}
\newcommand{\ConstantTok}[1]{\textcolor[rgb]{0.00,0.00,0.00}{#1}}
\newcommand{\ControlFlowTok}[1]{\textcolor[rgb]{0.13,0.29,0.53}{\textbf{#1}}}
\newcommand{\DataTypeTok}[1]{\textcolor[rgb]{0.13,0.29,0.53}{#1}}
\newcommand{\DecValTok}[1]{\textcolor[rgb]{0.00,0.00,0.81}{#1}}
\newcommand{\DocumentationTok}[1]{\textcolor[rgb]{0.56,0.35,0.01}{\textbf{\textit{#1}}}}
\newcommand{\ErrorTok}[1]{\textcolor[rgb]{0.64,0.00,0.00}{\textbf{#1}}}
\newcommand{\ExtensionTok}[1]{#1}
\newcommand{\FloatTok}[1]{\textcolor[rgb]{0.00,0.00,0.81}{#1}}
\newcommand{\FunctionTok}[1]{\textcolor[rgb]{0.00,0.00,0.00}{#1}}
\newcommand{\ImportTok}[1]{#1}
\newcommand{\InformationTok}[1]{\textcolor[rgb]{0.56,0.35,0.01}{\textbf{\textit{#1}}}}
\newcommand{\KeywordTok}[1]{\textcolor[rgb]{0.13,0.29,0.53}{\textbf{#1}}}
\newcommand{\NormalTok}[1]{#1}
\newcommand{\OperatorTok}[1]{\textcolor[rgb]{0.81,0.36,0.00}{\textbf{#1}}}
\newcommand{\OtherTok}[1]{\textcolor[rgb]{0.56,0.35,0.01}{#1}}
\newcommand{\PreprocessorTok}[1]{\textcolor[rgb]{0.56,0.35,0.01}{\textit{#1}}}
\newcommand{\RegionMarkerTok}[1]{#1}
\newcommand{\SpecialCharTok}[1]{\textcolor[rgb]{0.00,0.00,0.00}{#1}}
\newcommand{\SpecialStringTok}[1]{\textcolor[rgb]{0.31,0.60,0.02}{#1}}
\newcommand{\StringTok}[1]{\textcolor[rgb]{0.31,0.60,0.02}{#1}}
\newcommand{\VariableTok}[1]{\textcolor[rgb]{0.00,0.00,0.00}{#1}}
\newcommand{\VerbatimStringTok}[1]{\textcolor[rgb]{0.31,0.60,0.02}{#1}}
\newcommand{\WarningTok}[1]{\textcolor[rgb]{0.56,0.35,0.01}{\textbf{\textit{#1}}}}
\usepackage{graphicx}
\makeatletter
\def\maxwidth{\ifdim\Gin@nat@width>\linewidth\linewidth\else\Gin@nat@width\fi}
\def\maxheight{\ifdim\Gin@nat@height>\textheight\textheight\else\Gin@nat@height\fi}
\makeatother
% Scale images if necessary, so that they will not overflow the page
% margins by default, and it is still possible to overwrite the defaults
% using explicit options in \includegraphics[width, height, ...]{}
\setkeys{Gin}{width=\maxwidth,height=\maxheight,keepaspectratio}
\ifxetex
  \usepackage[setpagesize=false, % page size defined by xetex
              unicode=false, % unicode breaks when used with xetex
              xetex]{hyperref}
\else
  \usepackage[unicode=true]{hyperref}
\fi
\hypersetup{breaklinks=true,
            bookmarks=true,
            pdfauthor={Alexander Langnau, Öcal Kaptan, Sunyoung Ji},
            pdftitle={A Functional Approach to (Parallelised) Monte Carlo Simulation},
            colorlinks=true,
            citecolor=blue,
            urlcolor=blue,
            linkcolor=magenta,
            pdfborder={0 0 0}}
\urlstyle{same}  % don't use monospace font for urls
\setlength{\parindent}{0pt}
\setlength{\parskip}{6pt plus 2pt minus 1pt}
\setlength{\emergencystretch}{3em}  % prevent overfull lines
\setcounter{secnumdepth}{5}

%%% Use protect on footnotes to avoid problems with footnotes in titles
\let\rmarkdownfootnote\footnote%
\def\footnote{\protect\rmarkdownfootnote}

%%% Change title format to be more compact
\usepackage{titling}

% Create subtitle command for use in maketitle
\newcommand{\subtitle}[1]{
  \posttitle{
    \begin{center}\large#1\end{center}
    }
}

\setlength{\droptitle}{-2em}
  \title{A Functional Approach to (Parallelised) Monte Carlo Simulation}
  \pretitle{\vspace{\droptitle}\centering\huge}
  \posttitle{\par}
\subtitle{Advanced R for Econometricians}
  \author{Alexander Langnau, Öcal Kaptan, Sunyoung Ji}
  \preauthor{\centering\large\emph}
  \postauthor{\par}
  \predate{\centering\large\emph}
  \postdate{\par}
  \date{today}


%% linespread settings

\usepackage{setspace}

\onehalfspacing

% Language Setup

\usepackage{ifthen}
\usepackage{iflang}
\usepackage[super]{nth}

\ifthenelse{\equal{english}{german}}{
  \usepackage[ngerman]{babel}
  }{
  \usepackage[english]{babel}
  }

\begin{document}

\newgeometry{left=2cm,right=1cm,bottom=2cm,top=2cm}

\begin{titlepage}
  \noindent\begin{minipage}{0.6\textwidth}
	  \IfLanguageName{english}{University of Duisburg-Essen}{Universität Duisburg-Essen}\\
	  \IfLanguageName{english}{Faculty of Business Administration and Economics}{Fakultät für Wirtschaftswissensschaften}\\
	  \IfLanguageName{english}{Chair of Econometrics}{Lehrstuhl für Ökonometrie}\\
  \end{minipage}
	\begin{minipage}{0.4\textwidth}
	  \begin{flushright}
  	  \vspace{-0.5cm}
      \IfLanguageName{english}{\includegraphics*[width=5cm]{Includes/duelogo_en.png}}{\includegraphics*[width=5cm]{Includes/duelogo_de.png}}
	  \end{flushright}
	\end{minipage}
  \\
  \vspace{1.5cm}
  \begin{center}
  \huge{A Functional Approach to (Parallelised) Monte Carlo
Simulation}\\
  \vspace{.25cm}
  \Large{Advanced R for Econometricians}\\
  \vspace{0.5cm}
  \large{Final Project}\\
  \vspace{1cm}
  \large{
  \IfLanguageName{english}{Submitted to the Faculty of \\ Business Administration and Economics \\at the \\University of Duisburg-Essen}{Vorgelegt der \\Fakultät für Wirtschaftswissenschaften der \\ Universität Duisburg-Essen}\\}
  \vspace{0.75cm}
  \large{\IfLanguageName{english}{from:}{von:}}\\
  \vspace{0.5cm}
  Alexander Langnau, Öcal Kaptan, Sunyoung Ji\\
  \end{center}
  \vspace{4cm}

  \noindent\begin{minipage}[t]{0.5\textwidth}
  \IfLanguageName{english}{Matriculation Number:}{Matrikelnummer}
  \end{minipage}
  \begin{minipage}[t]{0.7\textwidth}
  \hspace{1cm}232907, 230914, 229979
  \end{minipage}

  \noindent\begin{minipage}[t]{0.5\textwidth}
  \IfLanguageName{english}{Study Path:}{Studienfach:}
  \end{minipage}
  \begin{minipage}[t]{0.7\textwidth}
  \hspace{1cm}M.Sc. Econometircs
  \end{minipage}

  \noindent\begin{minipage}[t]{0.5\textwidth}
  \IfLanguageName{english}{Reviewer:}{Erstgutachter:}
  \end{minipage}
  \begin{minipage}[t]{0.7\textwidth}
  \hspace{1cm}Prof.~Dr.~Christoph Hanck
  \end{minipage}

  \noindent\begin{minipage}[t]{0.5\textwidth}
  \IfLanguageName{english}{Secondary Reviewer:}{Zweitgutachter}
  \end{minipage}
  \begin{minipage}[t]{0.7\textwidth}
  \hspace{1cm}M.Sc. Martin C. Arnold, M.Sc. Jens Klenke
  \end{minipage}

  \noindent\begin{minipage}[t]{0.5\textwidth}
  Semester:
  \end{minipage}
  \begin{minipage}[t]{0.7\textwidth}
  \hspace{1cm}\IfLanguageName{english}{\nth{1} Semester}{1. Fachsemester}
  \end{minipage}

  \noindent\begin{minipage}[t]{0.5\textwidth}
  \IfLanguageName{english}{Graduation (est.):}{Vsl. Studienabschluss:}
  \end{minipage}
  \begin{minipage}[t]{0.7\textwidth}
  \hspace{1cm}Summer Term 2022
  \end{minipage}

  \noindent\begin{minipage}[t]{0.5\textwidth}
  \IfLanguageName{english}{Deadline:}{Abgabefrist:}
  \end{minipage}
  \begin{minipage}[t]{0.7\textwidth}
  \hspace{1cm}09. 09. 2022
  \end{minipage}

\end{titlepage}

% Ends the declared geometry for the titlepage
\restoregeometry


\pagenumbering{Roman} 
{
\hypersetup{linkcolor=black}
\setcounter{tocdepth}{3}
\tableofcontents
}
\newpage
\listoftables
\newpage
\listoffigures
\newpage
\pagenumbering{arabic} 
\hypertarget{introduction}{%
\section{Introduction}\label{introduction}}

Monte Carlo, named after a casino in Monaco, simulates complex
probabilistic events using simple random events, such as the tossing of
a pair of dice to simulate the casino's overall business model. In Monte
Carlo computing, a pseudo-random number generator is repeatedly called
which returns a real number in {[}0, 1{]}, and the results are used to
generate a distribution of samples that is a fair representation of the
target probability distribution under study.
\autocite[Adrian\_2022]{Barbu} Monte Carlo Method is combined with
programming in modern research and contributes to various studies.

Monte Carlo simulations are and will stay an important method in the
tool box of any econometrician, statistican or data scientist. Since
these simulations may be needed on a regular basis or are run over a
complex set of functions and parameters, its time well spend to
implement some tools, that allow the user to easily create a variety of
different Monte Carlo studies.

This paper was the final project of the course ``Advanced R for
econometricians'' at the chair of econometrics at university Duisburg
Essen. The goal is to use a functional programming aproach to create a
collection of different wrapper functions in R, that - providing a
convenient interface for Monte Carlo Simulations - create a paramter
grid - iterate homogenous function calls over the parameter grid -
provides an informative summary of the simulation results - can be
visualized by ggplot-methods - offers the possibility to use
parallelised processing (using \texttt{furrr} package)

A functional programming approach is well suited to implement the
different steps. The structure of this paper underlying code in general
follows this approach:

In chapter xyz we introduce different functions, that each specifically
solve the task of the bullet points mentioned above. In the beginning
we´ll underline the motivation and problem behind each function and
showcase the code.

At the end of each section we provide a minimal working example, that
illustrates the function and its output. We tried to implement in a way,
that the function works for as much cases, as possible. If there are
some restrictions regarding the usage of those functions, we´ll briefly
discuss them as well.

\hypertarget{preprocess-helper-functions}{%
\section{Preprocess / Helper
functions}\label{preprocess-helper-functions}}

\hypertarget{function-for-creating-grid}{%
\subsection{Function for creating
grid}\label{function-for-creating-grid}}

\texttt{create\_grid} is the function hat creates a parameter grid with
all permutations of the given parameters. This is necessary to try all
possible combinations to find the optimal parameters. This function
tunes parameters to improve performance of Monte Carlo Simulation
function.

\begin{Shaded}
\begin{Highlighting}[]
\NormalTok{create\_grid }\OtherTok{\textless{}{-}} \ControlFlowTok{function}\NormalTok{(parameters, nrep)\{}
\NormalTok{  input }\OtherTok{\textless{}{-}}\NormalTok{ parameters}
\NormalTok{  storage }\OtherTok{\textless{}{-}} \FunctionTok{list}\NormalTok{()}
\NormalTok{  name\_vec }\OtherTok{\textless{}{-}} \FunctionTok{c}\NormalTok{()}
  
  \ControlFlowTok{for}\NormalTok{(i }\ControlFlowTok{in} \DecValTok{1}\SpecialCharTok{:}\FunctionTok{length}\NormalTok{(input))\{ }\CommentTok{\#1:3}
\NormalTok{    a }\OtherTok{\textless{}{-}} \FunctionTok{as.numeric}\NormalTok{(input[[i]][[}\DecValTok{2}\NormalTok{]])}
\NormalTok{    b }\OtherTok{\textless{}{-}} \FunctionTok{as.numeric}\NormalTok{(input[[i]][[}\DecValTok{3}\NormalTok{]])}
\NormalTok{    c }\OtherTok{\textless{}{-}} \FunctionTok{as.numeric}\NormalTok{(input[[i]][[}\DecValTok{4}\NormalTok{]])}
\NormalTok{    output }\OtherTok{\textless{}{-}} \FunctionTok{seq}\NormalTok{(}\AttributeTok{from=}\NormalTok{a, }\AttributeTok{to=}\NormalTok{b, }\AttributeTok{by=}\NormalTok{c)}
\NormalTok{    storage[[i]] }\OtherTok{\textless{}{-}}\NormalTok{  output}
\NormalTok{    name\_vec[i] }\OtherTok{\textless{}{-}}\NormalTok{ input[[i]][[}\DecValTok{1}\NormalTok{]]}
\NormalTok{  \}}
  
\NormalTok{  grid }\OtherTok{\textless{}{-}} \FunctionTok{expand\_grid}\NormalTok{(}\FunctionTok{unlist}\NormalTok{(storage[}\DecValTok{1}\NormalTok{])}
\NormalTok{                      , }\FunctionTok{unlist}\NormalTok{(storage[}\DecValTok{2}\NormalTok{])}
\NormalTok{                      , }\FunctionTok{unlist}\NormalTok{(storage[}\DecValTok{3}\NormalTok{])}
\NormalTok{                      , }\FunctionTok{unlist}\NormalTok{(storage[}\DecValTok{4}\NormalTok{])}
\NormalTok{                      , }\FunctionTok{unlist}\NormalTok{(storage[}\DecValTok{5}\NormalTok{])}
\NormalTok{                      , }\FunctionTok{c}\NormalTok{(}\DecValTok{1}\SpecialCharTok{:}\NormalTok{nrep))}
  
  \FunctionTok{names}\NormalTok{(grid) }\OtherTok{\textless{}{-}} \FunctionTok{c}\NormalTok{(name\_vec, }\StringTok{"rep"}\NormalTok{)}
  
  \FunctionTok{return}\NormalTok{(grid)}
\NormalTok{\}}
\end{Highlighting}
\end{Shaded}

\texttt{create\_grid()} Example:

\begin{Shaded}
\begin{Highlighting}[]
\CommentTok{\#One parameter (works)}
\NormalTok{param\_list1 }\OtherTok{\textless{}{-}} \FunctionTok{list}\NormalTok{(}\FunctionTok{c}\NormalTok{(}\StringTok{"n"}\NormalTok{, }\DecValTok{10}\NormalTok{, }\DecValTok{20}\NormalTok{, }\DecValTok{10}\NormalTok{))}
\FunctionTok{tail}\NormalTok{(}\FunctionTok{create\_grid}\NormalTok{(param\_list1, }\AttributeTok{nrep=}\DecValTok{10}\NormalTok{), }\DecValTok{2}\NormalTok{)}
\end{Highlighting}
\end{Shaded}

\begin{verbatim}
## # A tibble: 2 x 2
##       n   rep
##   <dbl> <int>
## 1    20     9
## 2    20    10
\end{verbatim}

\begin{Shaded}
\begin{Highlighting}[]
\FunctionTok{tail}\NormalTok{(}\FunctionTok{create\_grid}\NormalTok{(param\_list1, }\AttributeTok{nrep=}\DecValTok{1}\NormalTok{), }\DecValTok{2}\NormalTok{)}
\end{Highlighting}
\end{Shaded}

\begin{verbatim}
## # A tibble: 2 x 2
##       n   rep
##   <dbl> <int>
## 1    10     1
## 2    20     1
\end{verbatim}

\begin{Shaded}
\begin{Highlighting}[]
\CommentTok{\#two parameter (works)}
\NormalTok{param\_list2 }\OtherTok{\textless{}{-}} \FunctionTok{list}\NormalTok{(}\FunctionTok{c}\NormalTok{(}\StringTok{"n"}\NormalTok{, }\DecValTok{10}\NormalTok{, }\DecValTok{20}\NormalTok{, }\DecValTok{10}\NormalTok{)}
\NormalTok{                    ,}\FunctionTok{c}\NormalTok{(}\StringTok{"mu"}\NormalTok{, }\DecValTok{0}\NormalTok{, }\DecValTok{1}\NormalTok{, }\FloatTok{0.25}\NormalTok{))}
\FunctionTok{tail}\NormalTok{(}\FunctionTok{create\_grid}\NormalTok{(param\_list1, }\AttributeTok{nrep=}\DecValTok{10}\NormalTok{), }\DecValTok{2}\NormalTok{)}
\end{Highlighting}
\end{Shaded}

\begin{verbatim}
## # A tibble: 2 x 2
##       n   rep
##   <dbl> <int>
## 1    20     9
## 2    20    10
\end{verbatim}

\begin{Shaded}
\begin{Highlighting}[]
\CommentTok{\#three parameters (works)}
\NormalTok{param\_list3 }\OtherTok{\textless{}{-}} \FunctionTok{list}\NormalTok{(}\FunctionTok{c}\NormalTok{(}\StringTok{"n"}\NormalTok{, }\DecValTok{10}\NormalTok{, }\DecValTok{20}\NormalTok{, }\DecValTok{10}\NormalTok{)}
\NormalTok{                    ,}\FunctionTok{c}\NormalTok{(}\StringTok{"mu"}\NormalTok{, }\DecValTok{0}\NormalTok{, }\DecValTok{1}\NormalTok{, }\FloatTok{0.25}\NormalTok{)}
\NormalTok{                    ,}\FunctionTok{c}\NormalTok{(}\StringTok{"sd"}\NormalTok{, }\DecValTok{0}\NormalTok{, }\FloatTok{0.3}\NormalTok{, }\FloatTok{0.1}\NormalTok{))}
\FunctionTok{tail}\NormalTok{(}\FunctionTok{create\_grid}\NormalTok{(param\_list3, }\AttributeTok{nrep=}\DecValTok{10}\NormalTok{), }\DecValTok{2}\NormalTok{)}
\end{Highlighting}
\end{Shaded}

\begin{verbatim}
## # A tibble: 2 x 4
##       n    mu    sd   rep
##   <dbl> <dbl> <dbl> <int>
## 1    20     1   0.3     9
## 2    20     1   0.3    10
\end{verbatim}

\begin{Shaded}
\begin{Highlighting}[]
\CommentTok{\#four parameters (works)}
\NormalTok{param\_list4 }\OtherTok{\textless{}{-}} \FunctionTok{list}\NormalTok{(}\FunctionTok{c}\NormalTok{(}\StringTok{"n"}\NormalTok{, }\DecValTok{10}\NormalTok{, }\DecValTok{20}\NormalTok{, }\DecValTok{10}\NormalTok{)}
\NormalTok{                    ,}\FunctionTok{c}\NormalTok{(}\StringTok{"mu"}\NormalTok{, }\DecValTok{0}\NormalTok{, }\DecValTok{1}\NormalTok{, }\FloatTok{0.25}\NormalTok{)}
\NormalTok{                    ,}\FunctionTok{c}\NormalTok{(}\StringTok{"sd"}\NormalTok{, }\DecValTok{0}\NormalTok{, }\FloatTok{0.3}\NormalTok{, }\FloatTok{0.1}\NormalTok{)}
\NormalTok{                    ,}\FunctionTok{c}\NormalTok{(}\StringTok{"gender"}\NormalTok{, }\DecValTok{0}\NormalTok{, }\DecValTok{1}\NormalTok{, }\DecValTok{1}\NormalTok{))}

\FunctionTok{tail}\NormalTok{(}\FunctionTok{create\_grid}\NormalTok{(param\_list4, }\AttributeTok{nrep=}\DecValTok{5}\NormalTok{),}\DecValTok{2}\NormalTok{)}
\end{Highlighting}
\end{Shaded}

\begin{verbatim}
## # A tibble: 2 x 5
##       n    mu    sd gender   rep
##   <dbl> <dbl> <dbl>  <dbl> <int>
## 1    20     1   0.3      1     4
## 2    20     1   0.3      1     5
\end{verbatim}

\begin{Shaded}
\begin{Highlighting}[]
\NormalTok{grid\_4 }\OtherTok{\textless{}{-}} \FunctionTok{create\_grid}\NormalTok{(param\_list4, }\AttributeTok{nrep=}\DecValTok{50}\NormalTok{)}
\FunctionTok{tail}\NormalTok{(grid\_4,}\DecValTok{2}\NormalTok{)}
\end{Highlighting}
\end{Shaded}

\begin{verbatim}
## # A tibble: 2 x 5
##       n    mu    sd gender   rep
##   <dbl> <dbl> <dbl>  <dbl> <int>
## 1    20     1   0.3      1    49
## 2    20     1   0.3      1    50
\end{verbatim}

\hypertarget{data-generation-function}{%
\subsection{Data generation function}\label{data-generation-function}}

\texttt{data\_generation} allows users to flexibly change data while
keeping the summary statistics and to choose the number of inputs by
using different \texttt{purrr} mapping functions: map, map2, and pmap
for a input, two inputs, and p inputs respectively.

In the function below, \texttt{simulation} means a distribution of data,
and \texttt{grid} is a list of parameters.

\begin{Shaded}
\begin{Highlighting}[]
\NormalTok{data\_generation }\OtherTok{\textless{}{-}} \ControlFlowTok{function}\NormalTok{(simulation, grid)\{ }
  \CommentTok{\#this is for use inside the function}
  
  \ControlFlowTok{if}\NormalTok{(}\FunctionTok{ncol}\NormalTok{(grid)}\SpecialCharTok{==}\DecValTok{2}\NormalTok{)\{}
\NormalTok{    var1 }\OtherTok{\textless{}{-}} \FunctionTok{c}\NormalTok{(}\FunctionTok{unlist}\NormalTok{(grid[,}\DecValTok{1}\NormalTok{]))}
\NormalTok{    data }\OtherTok{\textless{}{-}} \FunctionTok{map}\NormalTok{(var1, simulation) }
    \CommentTok{\#different purrr{-}functions depending on how many input variables we use}
\NormalTok{  \}}
  
  \ControlFlowTok{if}\NormalTok{(}\FunctionTok{ncol}\NormalTok{(grid)}\SpecialCharTok{==}\DecValTok{3}\NormalTok{)\{}
\NormalTok{    var1 }\OtherTok{\textless{}{-}} \FunctionTok{c}\NormalTok{(}\FunctionTok{unlist}\NormalTok{(grid[,}\DecValTok{1}\NormalTok{]))}
\NormalTok{    var2 }\OtherTok{\textless{}{-}} \FunctionTok{c}\NormalTok{(}\FunctionTok{unlist}\NormalTok{(grid[,}\DecValTok{2}\NormalTok{]))}
\NormalTok{    data }\OtherTok{\textless{}{-}} \FunctionTok{map2}\NormalTok{(var1, var2, simulation)}
\NormalTok{  \} }
  
  \ControlFlowTok{if}\NormalTok{(}\FunctionTok{ncol}\NormalTok{(grid)}\SpecialCharTok{==}\DecValTok{4}\NormalTok{)\{ }
\NormalTok{    var1 }\OtherTok{\textless{}{-}} \FunctionTok{c}\NormalTok{(}\FunctionTok{unlist}\NormalTok{(grid[,}\DecValTok{1}\NormalTok{]))}
\NormalTok{    var2 }\OtherTok{\textless{}{-}} \FunctionTok{c}\NormalTok{(}\FunctionTok{unlist}\NormalTok{(grid[,}\DecValTok{2}\NormalTok{]))}
\NormalTok{    var3 }\OtherTok{\textless{}{-}} \FunctionTok{c}\NormalTok{(}\FunctionTok{unlist}\NormalTok{(grid[,}\DecValTok{3}\NormalTok{]))}
\NormalTok{    list1 }\OtherTok{\textless{}{-}} \FunctionTok{list}\NormalTok{(var1,var2,var3)}
\NormalTok{    data }\OtherTok{\textless{}{-}} \FunctionTok{pmap}\NormalTok{(list1, }\AttributeTok{.f=}\NormalTok{simulation)}
\NormalTok{  \} }
  
  \FunctionTok{return}\NormalTok{(data)}
\NormalTok{\}}
\end{Highlighting}
\end{Shaded}

\texttt{data\_generation()} Example:

\begin{Shaded}
\begin{Highlighting}[]
\NormalTok{grid1 }\OtherTok{\textless{}{-}} \FunctionTok{create\_grid}\NormalTok{(param\_list1, }\AttributeTok{nrep=}\DecValTok{3}\NormalTok{)}
\FunctionTok{tail}\NormalTok{(}\FunctionTok{data\_generation}\NormalTok{(}\AttributeTok{simulation=}\NormalTok{rnorm, }\AttributeTok{grid=}\NormalTok{grid1),}\DecValTok{1}\NormalTok{)}
\end{Highlighting}
\end{Shaded}

\begin{verbatim}
## $n6
##  [1] -0.491031166 -2.309168876  1.005738524 -0.709200763 -0.688008616
##  [6]  1.025571370 -0.284773007 -1.220717712  0.181303480 -0.138891362
## [11]  0.005764186  0.385280401 -0.370660032  0.644376549 -0.220486562
## [16]  0.331781964  1.096839013  0.435181491 -0.325931586  1.148807618
\end{verbatim}

\begin{Shaded}
\begin{Highlighting}[]
\NormalTok{grid2 }\OtherTok{\textless{}{-}} \FunctionTok{create\_grid}\NormalTok{(param\_list2, }\AttributeTok{nrep=}\DecValTok{3}\NormalTok{)}
\FunctionTok{tail}\NormalTok{(}\FunctionTok{data\_generation}\NormalTok{(}\AttributeTok{simulation=}\NormalTok{rnorm, }\AttributeTok{grid=}\NormalTok{grid2),}\DecValTok{1}\NormalTok{)}
\end{Highlighting}
\end{Shaded}

\begin{verbatim}
## $n30
##  [1]  1.9672673  0.8917199  0.3015793  0.7240548  2.1146485  1.5500440
##  [7]  2.2366758  1.1390979  1.4102751  0.4415431  1.6053707  0.4936665
## [13] -0.4205655  1.1279930  2.9458512  1.8009143  2.1652534  1.3588557
## [19]  0.3914428  0.7977591
\end{verbatim}

Users can apply many distributions such as normal, uniform, poisson
distributions by putting existing functions in r as \texttt{simulation}.

\begin{Shaded}
\begin{Highlighting}[]
\CommentTok{\# Application to Uniform distribution}
\NormalTok{param\_list\_runif }\OtherTok{\textless{}{-}} \FunctionTok{list}\NormalTok{(}\FunctionTok{c}\NormalTok{(}\StringTok{"n"}\NormalTok{, }\DecValTok{10}\NormalTok{, }\DecValTok{30}\NormalTok{, }\DecValTok{10}\NormalTok{)}
\NormalTok{                         ,}\FunctionTok{c}\NormalTok{(}\StringTok{"min"}\NormalTok{, }\DecValTok{0}\NormalTok{, }\DecValTok{0}\NormalTok{, }\DecValTok{0}\NormalTok{)}
\NormalTok{                         ,}\FunctionTok{c}\NormalTok{(}\StringTok{"max"}\NormalTok{, }\DecValTok{1}\NormalTok{, }\DecValTok{1}\NormalTok{, }\DecValTok{0}\NormalTok{))}


\NormalTok{grid\_unif }\OtherTok{\textless{}{-}} \FunctionTok{create\_grid}\NormalTok{(param\_list\_runif, }\AttributeTok{nrep=}\DecValTok{3}\NormalTok{)}
\FunctionTok{tail}\NormalTok{(}\FunctionTok{data\_generation}\NormalTok{(}\AttributeTok{simulation=}\NormalTok{runif, }\AttributeTok{grid=}\NormalTok{grid\_unif),}\DecValTok{1}\NormalTok{)}
\end{Highlighting}
\end{Shaded}

\begin{verbatim}
## $n9
##  [1] 0.004638151 0.277560080 0.325203143 0.588706277 0.249684701 0.043117281
##  [7] 0.110678788 0.703753812 0.939021239 0.311169018 0.078492930 0.321744091
## [13] 0.624905537 0.440241850 0.801345301 0.279283805 0.570713193 0.042128012
## [19] 0.190717455 0.727086471 0.826690050 0.510721075 0.567726166 0.001155820
## [25] 0.143778103 0.865967083 0.082561061 0.244570682 0.981543157 0.577581279
\end{verbatim}

\begin{Shaded}
\begin{Highlighting}[]
\CommentTok{\# Application to Poisson distribution}

\NormalTok{param\_list\_rpois }\OtherTok{\textless{}{-}} \FunctionTok{list}\NormalTok{(}\FunctionTok{c}\NormalTok{(}\StringTok{"n"}\NormalTok{, }\DecValTok{10}\NormalTok{, }\DecValTok{30}\NormalTok{, }\DecValTok{10}\NormalTok{)}
\NormalTok{                         , }\FunctionTok{c}\NormalTok{(}\StringTok{"lambda"}\NormalTok{, }\DecValTok{0}\NormalTok{, }\DecValTok{10}\NormalTok{, }\DecValTok{1}\NormalTok{))}

\NormalTok{grid\_pois }\OtherTok{\textless{}{-}} \FunctionTok{create\_grid}\NormalTok{(param\_list\_rpois, }\AttributeTok{nrep=}\DecValTok{3}\NormalTok{)}
\FunctionTok{tail}\NormalTok{(grid\_pois,}\DecValTok{2}\NormalTok{) }\CommentTok{\# nrow(grid\_pois) = 99}
\end{Highlighting}
\end{Shaded}

\begin{verbatim}
## # A tibble: 2 x 3
##       n lambda   rep
##   <dbl>  <dbl> <int>
## 1    30     10     2
## 2    30     10     3
\end{verbatim}

\begin{Shaded}
\begin{Highlighting}[]
\FunctionTok{tail}\NormalTok{(}\FunctionTok{data\_generation}\NormalTok{(}\AttributeTok{simulation=}\NormalTok{rpois, }\AttributeTok{grid=}\NormalTok{grid\_pois),}\DecValTok{1}\NormalTok{)}
\end{Highlighting}
\end{Shaded}

\begin{verbatim}
## $n99
##  [1]  8  8  8 12  7  6 10  9  5  8 19 12  7 12 13  7  3  9  7 15  6 13 11 15  8
## [26] 13  9  7  9  5
\end{verbatim}

\hypertarget{summary-function}{%
\subsection{Summary function}\label{summary-function}}

\texttt{summary\_function} offers summary statistics that users can
choose.

\begin{Shaded}
\begin{Highlighting}[]
\CommentTok{\#summary function for one input}
\NormalTok{summary\_function }\OtherTok{\textless{}{-}} \ControlFlowTok{function}\NormalTok{(sum\_fun, data\_input)\{}
  
\NormalTok{  count }\OtherTok{\textless{}{-}} \FunctionTok{length}\NormalTok{(data\_input)}
\NormalTok{  summary\_matrix }\OtherTok{\textless{}{-}} \FunctionTok{matrix}\NormalTok{(}\AttributeTok{nrow=}\NormalTok{count, }\AttributeTok{ncol=}\DecValTok{1}\NormalTok{)}
  
  \ControlFlowTok{for}\NormalTok{(i }\ControlFlowTok{in} \DecValTok{1}\SpecialCharTok{:}\NormalTok{count)\{}
\NormalTok{    input }\OtherTok{\textless{}{-}} \FunctionTok{list}\NormalTok{(data\_input[[i]])}
\NormalTok{    output }\OtherTok{\textless{}{-}} \FunctionTok{sapply}\NormalTok{(sum\_fun, do.call, input)}
\NormalTok{    summary\_matrix[i] }\OtherTok{\textless{}{-}}\NormalTok{ output}
\NormalTok{  \}}
  \CommentTok{\#output \textless{}{-} as.data.frame(summary\_matrix)}
  \CommentTok{\#names(output) \textless{}{-} sum\_fun}
  \FunctionTok{colnames}\NormalTok{(summary\_matrix) }\OtherTok{\textless{}{-}}\NormalTok{ sum\_fun}
  \FunctionTok{return}\NormalTok{(summary\_matrix)}
\NormalTok{\}}
\end{Highlighting}
\end{Shaded}

\texttt{summary\_function} Example:

\begin{Shaded}
\begin{Highlighting}[]
\NormalTok{grid\_test }\OtherTok{\textless{}{-}} \FunctionTok{create\_grid}\NormalTok{(param\_list3, }\AttributeTok{nrep=}\DecValTok{3}\NormalTok{)}
\NormalTok{test\_data }\OtherTok{\textless{}{-}} \FunctionTok{data\_generation}\NormalTok{(}\AttributeTok{simulation=}\NormalTok{rnorm, }\AttributeTok{grid=}\NormalTok{grid\_test)}
\FunctionTok{tail}\NormalTok{(}\FunctionTok{summary\_function}\NormalTok{(}\AttributeTok{sum\_fun=}\FunctionTok{list}\NormalTok{(}\StringTok{"mean"}\NormalTok{), }\AttributeTok{data\_input=}\NormalTok{test\_data),}\DecValTok{2}\NormalTok{)}
\end{Highlighting}
\end{Shaded}

\begin{verbatim}
##           mean
## [119,] 1.03361
## [120,] 1.01786
\end{verbatim}

\hypertarget{summary-array-funcation}{%
\subsection{Summary array funcation}\label{summary-array-funcation}}

The outcome of \texttt{create\_array\_function} illustrates the
combination of user defined grid and the summary statistics. This
function product dataframes with all permutations and results that
allow, thus users can look any possible parameter regarding specific
grid.

\begin{Shaded}
\begin{Highlighting}[]
\NormalTok{create\_array\_function }\OtherTok{\textless{}{-}} \ControlFlowTok{function}\NormalTok{(comb, parameters, nrep)\{}
\NormalTok{  storage }\OtherTok{\textless{}{-}} \FunctionTok{list}\NormalTok{()}
\NormalTok{  name\_vec }\OtherTok{\textless{}{-}} \FunctionTok{c}\NormalTok{()}
  
  \ControlFlowTok{for}\NormalTok{(i }\ControlFlowTok{in} \DecValTok{1}\SpecialCharTok{:}\FunctionTok{length}\NormalTok{(parameters))\{ }
    \CommentTok{\#this creates the sequences of parameters}
\NormalTok{    a }\OtherTok{\textless{}{-}} \FunctionTok{as.numeric}\NormalTok{(parameters[[i]][[}\DecValTok{2}\NormalTok{]])}
\NormalTok{    b }\OtherTok{\textless{}{-}} \FunctionTok{as.numeric}\NormalTok{(parameters[[i]][[}\DecValTok{3}\NormalTok{]])}
\NormalTok{    c }\OtherTok{\textless{}{-}} \FunctionTok{as.numeric}\NormalTok{(parameters[[i]][[}\DecValTok{4}\NormalTok{]])}
\NormalTok{    output }\OtherTok{\textless{}{-}} \FunctionTok{seq}\NormalTok{(}\AttributeTok{from=}\NormalTok{a, }\AttributeTok{to=}\NormalTok{b, }\AttributeTok{by=}\NormalTok{c)}
\NormalTok{    storage[[i]] }\OtherTok{\textless{}{-}}\NormalTok{  output}
\NormalTok{    name\_vec[i] }\OtherTok{\textless{}{-}}\NormalTok{ parameters[[i]][[}\DecValTok{1}\NormalTok{]] }
    \CommentTok{\#this just stores the names of the variables}
\NormalTok{  \}}
  
  
\NormalTok{  matrix.numeration }\OtherTok{\textless{}{-}}  \FunctionTok{paste}\NormalTok{(}\StringTok{"rep"}\NormalTok{,}\StringTok{"="}\NormalTok{, }\DecValTok{1}\SpecialCharTok{:}\NormalTok{nrep, }\AttributeTok{sep =} \StringTok{""}\NormalTok{)}
  
  \ControlFlowTok{if}\NormalTok{(}\FunctionTok{length}\NormalTok{(parameters)}\SpecialCharTok{==}\DecValTok{1}\NormalTok{)\{}
\NormalTok{    comb\_ordered }\OtherTok{\textless{}{-}}\NormalTok{  comb }\SpecialCharTok{\%\textgreater{}\%} \FunctionTok{arrange}\NormalTok{(comb[,}\DecValTok{2}\NormalTok{])}
\NormalTok{    seq1 }\OtherTok{\textless{}{-}} \FunctionTok{c}\NormalTok{(}\FunctionTok{unlist}\NormalTok{(storage[}\DecValTok{1}\NormalTok{]))}
    
\NormalTok{    row.names }\OtherTok{\textless{}{-}} \FunctionTok{paste}\NormalTok{(name\_vec[}\DecValTok{1}\NormalTok{],}\StringTok{"="}\NormalTok{,seq1, }\AttributeTok{sep =} \StringTok{""}\NormalTok{)}
    
\NormalTok{    dimension\_array }\OtherTok{\textless{}{-}} \FunctionTok{c}\NormalTok{(}\FunctionTok{length}\NormalTok{(seq1), nrep)}
\NormalTok{    dim\_names\_list }\OtherTok{\textless{}{-}} \FunctionTok{list}\NormalTok{(row.names, matrix.numeration)}
\NormalTok{  \}}
  
  \ControlFlowTok{if}\NormalTok{(}\FunctionTok{length}\NormalTok{(parameters)}\SpecialCharTok{==}\DecValTok{2}\NormalTok{)\{}
\NormalTok{    comb\_ordered }\OtherTok{\textless{}{-}}\NormalTok{  comb }\SpecialCharTok{\%\textgreater{}\%} \FunctionTok{arrange}\NormalTok{(comb[,}\DecValTok{2}\NormalTok{])  }\SpecialCharTok{\%\textgreater{}\%} \FunctionTok{arrange}\NormalTok{(comb[,}\DecValTok{3}\NormalTok{])}
\NormalTok{    seq1 }\OtherTok{\textless{}{-}} \FunctionTok{c}\NormalTok{(}\FunctionTok{unlist}\NormalTok{(storage[}\DecValTok{1}\NormalTok{]))}
\NormalTok{    seq2 }\OtherTok{\textless{}{-}} \FunctionTok{c}\NormalTok{(}\FunctionTok{unlist}\NormalTok{(storage[}\DecValTok{2}\NormalTok{]))}
    
\NormalTok{    row.names }\OtherTok{\textless{}{-}} \FunctionTok{paste}\NormalTok{(name\_vec[}\DecValTok{1}\NormalTok{],}\StringTok{"="}\NormalTok{,seq1, }\AttributeTok{sep =} \StringTok{""}\NormalTok{)}
\NormalTok{    column.names }\OtherTok{\textless{}{-}}  \FunctionTok{paste}\NormalTok{(name\_vec[}\DecValTok{2}\NormalTok{],}\StringTok{"="}\NormalTok{,seq2, }\AttributeTok{sep =} \StringTok{""}\NormalTok{)}
    
\NormalTok{    dimension\_array }\OtherTok{\textless{}{-}} \FunctionTok{c}\NormalTok{(}\FunctionTok{length}\NormalTok{(seq1), }\FunctionTok{length}\NormalTok{(seq2), nrep)}
\NormalTok{    dim\_names\_list }\OtherTok{\textless{}{-}} \FunctionTok{list}\NormalTok{(row.names, column.names, matrix.numeration)}
\NormalTok{  \}}
  
  \ControlFlowTok{if}\NormalTok{(}\FunctionTok{length}\NormalTok{(parameters)}\SpecialCharTok{==}\DecValTok{3}\NormalTok{)\{}
\NormalTok{    comb\_ordered }\OtherTok{\textless{}{-}}\NormalTok{  comb }\SpecialCharTok{\%\textgreater{}\%} \FunctionTok{arrange}\NormalTok{(comb[,}\DecValTok{2}\NormalTok{])  }\SpecialCharTok{\%\textgreater{}\%} 
      \FunctionTok{arrange}\NormalTok{(comb[,}\DecValTok{3}\NormalTok{]) }\SpecialCharTok{\%\textgreater{}\%} \FunctionTok{arrange}\NormalTok{(comb[,}\DecValTok{4}\NormalTok{]) }
\NormalTok{    seq1 }\OtherTok{\textless{}{-}} \FunctionTok{c}\NormalTok{(}\FunctionTok{unlist}\NormalTok{(storage[}\DecValTok{1}\NormalTok{]))}
\NormalTok{    seq2 }\OtherTok{\textless{}{-}} \FunctionTok{c}\NormalTok{(}\FunctionTok{unlist}\NormalTok{(storage[}\DecValTok{2}\NormalTok{]))}
\NormalTok{    seq3 }\OtherTok{\textless{}{-}} \FunctionTok{c}\NormalTok{(}\FunctionTok{unlist}\NormalTok{(storage[}\DecValTok{3}\NormalTok{]))}
    
\NormalTok{    row.names }\OtherTok{\textless{}{-}} \FunctionTok{paste}\NormalTok{(name\_vec[}\DecValTok{1}\NormalTok{],}\StringTok{"="}\NormalTok{,seq1, }\AttributeTok{sep =} \StringTok{""}\NormalTok{)}
\NormalTok{    column.names }\OtherTok{\textless{}{-}}  \FunctionTok{paste}\NormalTok{(name\_vec[}\DecValTok{2}\NormalTok{],}\StringTok{"="}\NormalTok{,seq2, }\AttributeTok{sep =} \StringTok{""}\NormalTok{)}
\NormalTok{    matrix.names1 }\OtherTok{\textless{}{-}}  \FunctionTok{paste}\NormalTok{(name\_vec[}\DecValTok{3}\NormalTok{],}\StringTok{"="}\NormalTok{,seq3, }\AttributeTok{sep =} \StringTok{""}\NormalTok{)}
    
\NormalTok{    dimension\_array }\OtherTok{\textless{}{-}} \FunctionTok{c}\NormalTok{(}\FunctionTok{length}\NormalTok{(seq1), }\FunctionTok{length}\NormalTok{(seq2), }\FunctionTok{length}\NormalTok{(seq3), nrep)}
\NormalTok{    dim\_names\_list }\OtherTok{\textless{}{-}} \FunctionTok{list}\NormalTok{(row.names, column.names, }
\NormalTok{                           matrix.names1, matrix.numeration)}
    
\NormalTok{  \}}
  
  
\NormalTok{  array1 }\OtherTok{\textless{}{-}} \FunctionTok{array}\NormalTok{(comb\_ordered[,}\FunctionTok{ncol}\NormalTok{(comb)] }
                  \CommentTok{\#change to automatically adjust dim}
\NormalTok{                  , }\AttributeTok{dim =}\NormalTok{ dimension\_array}
\NormalTok{                  , dim\_names\_list)}
  \FunctionTok{return}\NormalTok{(array1)}
\NormalTok{\}}
\end{Highlighting}
\end{Shaded}

\texttt{create\_array\_function} Example:

\begin{Shaded}
\begin{Highlighting}[]
\CommentTok{\# PREP }\AlertTok{TEST}\CommentTok{ \textasciigrave{}create\_array\_function\textasciigrave{}}
\NormalTok{main\_function\_array\_test }\OtherTok{\textless{}{-}}  \ControlFlowTok{function}\NormalTok{(parameters }\CommentTok{\#list of parameters}
\NormalTok{                                      , nrep }\CommentTok{\#number of repetitions}
\NormalTok{                                      , simulation }\CommentTok{\#data genereation}
\NormalTok{                                      , sum\_fun)\{ }\CommentTok{\#summary statistics}
  
\NormalTok{  grid }\OtherTok{\textless{}{-}} \FunctionTok{create\_grid}\NormalTok{(parameters, nrep) }\CommentTok{\#Step 1: create grid}
\NormalTok{  raw\_data }\OtherTok{\textless{}{-}} \FunctionTok{data\_generation}\NormalTok{(simulation, grid) }\CommentTok{\#Step 2: simlate data}
\NormalTok{  summary }\OtherTok{\textless{}{-}} \FunctionTok{summary\_function}\NormalTok{(sum\_fun, }\AttributeTok{data\_input=}\NormalTok{raw\_data) }\CommentTok{\#Step 3: Summary statistics}
\NormalTok{  comb }\OtherTok{\textless{}{-}} \FunctionTok{cbind}\NormalTok{(grid, summary) }\CommentTok{\#Step 4: Combine resuluts with parameters}
\NormalTok{  array\_1 }\OtherTok{\textless{}{-}} \FunctionTok{create\_array\_function}\NormalTok{(comb, parameters, nrep) }\CommentTok{\#Step 5: Create array}
  
  \FunctionTok{return}\NormalTok{(comb)}
\NormalTok{\}}

\NormalTok{param\_list3x }\OtherTok{\textless{}{-}} \FunctionTok{list}\NormalTok{(}\FunctionTok{c}\NormalTok{(}\StringTok{"n"}\NormalTok{, }\DecValTok{10}\NormalTok{, }\DecValTok{20}\NormalTok{, }\DecValTok{10}\NormalTok{)}
\NormalTok{                     ,}\FunctionTok{c}\NormalTok{(}\StringTok{"mu"}\NormalTok{, }\DecValTok{0}\NormalTok{, }\DecValTok{5}\NormalTok{, }\DecValTok{1}\NormalTok{)}
\NormalTok{                     ,}\FunctionTok{c}\NormalTok{(}\StringTok{"sd"}\NormalTok{, }\DecValTok{0}\NormalTok{, }\DecValTok{1}\NormalTok{, }\DecValTok{1}\NormalTok{))}

\NormalTok{comb1 }\OtherTok{\textless{}{-}} \FunctionTok{main\_function\_array\_test}\NormalTok{(}\AttributeTok{parameters=}\NormalTok{param\_list3x}
\NormalTok{                                  , }\AttributeTok{nrep =} \DecValTok{1}
\NormalTok{                                  , }\AttributeTok{simulation =}\NormalTok{ rnorm}
\NormalTok{                                  , }\AttributeTok{sum\_fun=}\StringTok{"mean"}\NormalTok{)}

\FunctionTok{head}\NormalTok{(comb1,}\DecValTok{2}\NormalTok{)}
\end{Highlighting}
\end{Shaded}

\begin{verbatim}
##    n mu sd rep       mean
## 1 10  0  0   1  0.0000000
## 2 10  0  1   1 -0.6031898
\end{verbatim}

\begin{Shaded}
\begin{Highlighting}[]
\FunctionTok{create\_array\_function}\NormalTok{(}\AttributeTok{comb=}\NormalTok{comb1, }\AttributeTok{parameters=}\NormalTok{param\_list3x, }\AttributeTok{nrep=}\DecValTok{1}\NormalTok{)}
\end{Highlighting}
\end{Shaded}

\begin{verbatim}
## , , sd=0, rep=1
## 
##      mu=0 mu=1 mu=2 mu=3 mu=4 mu=5
## n=10    0    1    2    3    4    5
## n=20    0    1    2    3    4    5
## 
## , , sd=1, rep=1
## 
##            mu=0      mu=1     mu=2     mu=3     mu=4     mu=5
## n=10 -0.6031898 1.1547493 1.768505 2.799209 4.297611 5.240045
## n=20 -0.1950611 0.7933902 1.609615 2.815089 4.066077 4.798390
\end{verbatim}

\hypertarget{monte-carlo-simulation-funcion}{%
\section{Monte Carlo Simulation
Funcion}\label{monte-carlo-simulation-funcion}}

\hypertarget{examples}{%
\section{Examples}\label{examples}}

\hypertarget{conclusion}{%
\section{Conclusion}\label{conclusion}}

The above section illustrates the power of our implemented model and
gives the fairly easy to use tool, that still allows for a variety of
different specifications in terms of used parameters, data generation
processes and summary functions. Researchers, who use Monte Carlo studys
on a regular basis, may save a lot of time using a tool like this in the
long run.

By nature, there may be cases, where our implementation doesnt satisfy
the needs of the user to the fullest, but for a wide variety of examples
we showed, that it worked well and served the goal that we aimed for.
Our functional programming approach allows for easy and flexible
adjustments in case the use of our functions should be expanded, f.e. if
a grid of more than 3 (or 4?) parameters is needed.

Theoretically, this work could be implemented as an R package to share
it with the R community. But since the \texttt{MonteCarlo()} function of
the \texttt{vigniette} package already provides a well working
alternative to our project besides some minor differences, there is
currently no need in doing that.

\pagebreak
\renewcommand*{\mkbibnamefamily}[1]{\textbf{#1}}
\renewcommand*{\mkbibnamegiven}[1]{\textbf{#1}}
\renewcommand*{\mkbibnameprefix}[1]{\textbf{#1}}
\renewcommand*{\mkbibnamesuffix}[1]{\textbf{#1}}
\printbibliography[title=References]

\newpage
\textbf{Eidesstattliche Versicherung}

\bigskip

Ich versichere an Eides statt durch meine Unterschrift, dass ich die vorstehende Arbeit selbständig und ohne fremde Hilfe angefertigt und alle Stellen, die ich wörtlich oder annähernd wörtlich aus Veröffentlichungen entnommen habe, als solche kenntlich gemacht habe, mich auch keiner anderen als der angegebenen Literatur oder sonstiger Hilfsmittel bedient habe. Die Arbeit hat in dieser oder ähnlicher Form noch keiner anderen Prüfungsbehörde vorgelegen.

\vspace{1cm}
\rule{0pt}{2\baselineskip} %
\par\noindent\makebox[2.25in]{\indent Essen, den \hrulefill} \hfill\makebox[2.25in]{\hrulefill}%
\par\noindent\makebox[2.25in][l]{} \hfill\makebox[2.25in][c]{Alexander
Langnau, Öcal Kaptan, Sunyoung Ji}%


\end{document}
