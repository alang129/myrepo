\documentclass[11pt,a4paper]{article}
\usepackage{lmodern}

\usepackage{amssymb,amsmath}
\usepackage{ifxetex,ifluatex}
\usepackage{fixltx2e} % provides \textsubscript
\ifnum 0\ifxetex 1\fi\ifluatex 1\fi=0 % if pdftex
  \usepackage[T1]{fontenc}
  \usepackage[utf8]{inputenc}
\else % if luatex or xelatex
  \ifxetex
    \usepackage{mathspec}
    \usepackage{xltxtra,xunicode}
  \else
    \usepackage{fontspec}
  \fi
  \defaultfontfeatures{Mapping=tex-text,Scale=MatchLowercase}
  \newcommand{\euro}{€}
\fi
% use upquote if available, for straight quotes in verbatim environments
\IfFileExists{upquote.sty}{\usepackage{upquote}}{}
% use microtype if available
\IfFileExists{microtype.sty}{%
\usepackage{microtype}
\UseMicrotypeSet[protrusion]{basicmath} % disable protrusion for tt fonts
}{}
\usepackage[lmargin=2.5cm,rmargin=2.5cm,tmargin=2.5cm,bmargin=2.5cm]{geometry}

% Figure Placement:
\usepackage{float}
\let\origfigure\figure
\let\endorigfigure\endfigure
\renewenvironment{figure}[1][2] {
    \expandafter\origfigure\expandafter[H]
} {
    \endorigfigure
}

%% citation setup

\usepackage{csquotes}

\usepackage[backend=biber, maxbibnames = 99, style = apa]{biblatex}
\setlength\bibitemsep{1.5\itemsep}
\bibliography{references.bib}
\usepackage{color}
\usepackage{fancyvrb}
\newcommand{\VerbBar}{|}
\newcommand{\VERB}{\Verb[commandchars=\\\{\}]}
\DefineVerbatimEnvironment{Highlighting}{Verbatim}{commandchars=\\\{\}}
% Add ',fontsize=\small' for more characters per line
\usepackage{framed}
\definecolor{shadecolor}{RGB}{248,248,248}
\newenvironment{Shaded}{\begin{snugshade}}{\end{snugshade}}
\newcommand{\AlertTok}[1]{\textcolor[rgb]{0.94,0.16,0.16}{#1}}
\newcommand{\AnnotationTok}[1]{\textcolor[rgb]{0.56,0.35,0.01}{\textbf{\textit{#1}}}}
\newcommand{\AttributeTok}[1]{\textcolor[rgb]{0.77,0.63,0.00}{#1}}
\newcommand{\BaseNTok}[1]{\textcolor[rgb]{0.00,0.00,0.81}{#1}}
\newcommand{\BuiltInTok}[1]{#1}
\newcommand{\CharTok}[1]{\textcolor[rgb]{0.31,0.60,0.02}{#1}}
\newcommand{\CommentTok}[1]{\textcolor[rgb]{0.56,0.35,0.01}{\textit{#1}}}
\newcommand{\CommentVarTok}[1]{\textcolor[rgb]{0.56,0.35,0.01}{\textbf{\textit{#1}}}}
\newcommand{\ConstantTok}[1]{\textcolor[rgb]{0.00,0.00,0.00}{#1}}
\newcommand{\ControlFlowTok}[1]{\textcolor[rgb]{0.13,0.29,0.53}{\textbf{#1}}}
\newcommand{\DataTypeTok}[1]{\textcolor[rgb]{0.13,0.29,0.53}{#1}}
\newcommand{\DecValTok}[1]{\textcolor[rgb]{0.00,0.00,0.81}{#1}}
\newcommand{\DocumentationTok}[1]{\textcolor[rgb]{0.56,0.35,0.01}{\textbf{\textit{#1}}}}
\newcommand{\ErrorTok}[1]{\textcolor[rgb]{0.64,0.00,0.00}{\textbf{#1}}}
\newcommand{\ExtensionTok}[1]{#1}
\newcommand{\FloatTok}[1]{\textcolor[rgb]{0.00,0.00,0.81}{#1}}
\newcommand{\FunctionTok}[1]{\textcolor[rgb]{0.00,0.00,0.00}{#1}}
\newcommand{\ImportTok}[1]{#1}
\newcommand{\InformationTok}[1]{\textcolor[rgb]{0.56,0.35,0.01}{\textbf{\textit{#1}}}}
\newcommand{\KeywordTok}[1]{\textcolor[rgb]{0.13,0.29,0.53}{\textbf{#1}}}
\newcommand{\NormalTok}[1]{#1}
\newcommand{\OperatorTok}[1]{\textcolor[rgb]{0.81,0.36,0.00}{\textbf{#1}}}
\newcommand{\OtherTok}[1]{\textcolor[rgb]{0.56,0.35,0.01}{#1}}
\newcommand{\PreprocessorTok}[1]{\textcolor[rgb]{0.56,0.35,0.01}{\textit{#1}}}
\newcommand{\RegionMarkerTok}[1]{#1}
\newcommand{\SpecialCharTok}[1]{\textcolor[rgb]{0.00,0.00,0.00}{#1}}
\newcommand{\SpecialStringTok}[1]{\textcolor[rgb]{0.31,0.60,0.02}{#1}}
\newcommand{\StringTok}[1]{\textcolor[rgb]{0.31,0.60,0.02}{#1}}
\newcommand{\VariableTok}[1]{\textcolor[rgb]{0.00,0.00,0.00}{#1}}
\newcommand{\VerbatimStringTok}[1]{\textcolor[rgb]{0.31,0.60,0.02}{#1}}
\newcommand{\WarningTok}[1]{\textcolor[rgb]{0.56,0.35,0.01}{\textbf{\textit{#1}}}}
\usepackage{longtable,booktabs}
\usepackage{graphicx}
\makeatletter
\def\maxwidth{\ifdim\Gin@nat@width>\linewidth\linewidth\else\Gin@nat@width\fi}
\def\maxheight{\ifdim\Gin@nat@height>\textheight\textheight\else\Gin@nat@height\fi}
\makeatother
% Scale images if necessary, so that they will not overflow the page
% margins by default, and it is still possible to overwrite the defaults
% using explicit options in \includegraphics[width, height, ...]{}
\setkeys{Gin}{width=\maxwidth,height=\maxheight,keepaspectratio}
\ifxetex
  \usepackage[setpagesize=false, % page size defined by xetex
              unicode=false, % unicode breaks when used with xetex
              xetex]{hyperref}
\else
  \usepackage[unicode=true]{hyperref}
\fi
\hypersetup{breaklinks=true,
            bookmarks=true,
            pdfauthor={Alexander Langnau, Öcal Kaptan, Sunyoung Ji},
            pdftitle={A Functional Approach to (Parallelised) Monte Carlo Simulation},
            colorlinks=true,
            citecolor=blue,
            urlcolor=blue,
            linkcolor=magenta,
            pdfborder={0 0 0}}
\urlstyle{same}  % don't use monospace font for urls
\setlength{\parindent}{0pt}
\setlength{\parskip}{6pt plus 2pt minus 1pt}
\setlength{\emergencystretch}{3em}  % prevent overfull lines
\setcounter{secnumdepth}{5}

%%% Use protect on footnotes to avoid problems with footnotes in titles
\let\rmarkdownfootnote\footnote%
\def\footnote{\protect\rmarkdownfootnote}

%%% Change title format to be more compact
\usepackage{titling}

% Create subtitle command for use in maketitle
\newcommand{\subtitle}[1]{
  \posttitle{
    \begin{center}\large#1\end{center}
    }
}

\setlength{\droptitle}{-2em}
  \title{A Functional Approach to (Parallelised) Monte Carlo Simulation}
  \pretitle{\vspace{\droptitle}\centering\huge}
  \posttitle{\par}
\subtitle{Advanced R for Econometricians}
  \author{Alexander Langnau, Öcal Kaptan, Sunyoung Ji}
  \preauthor{\centering\large\emph}
  \postauthor{\par}
  \predate{\centering\large\emph}
  \postdate{\par}
  \date{today}


%% linespread settings

\usepackage{setspace}

\onehalfspacing

% Language Setup

\usepackage{ifthen}
\usepackage{iflang}
\usepackage[super]{nth}

\ifthenelse{\equal{english}{german}}{
  \usepackage[ngerman]{babel}
  }{
  \usepackage[english]{babel}
  }

\begin{document}

\newgeometry{left=2cm,right=1cm,bottom=2cm,top=2cm}

\begin{titlepage}
  \noindent\begin{minipage}{0.6\textwidth}
	  \IfLanguageName{english}{University of Duisburg-Essen}{Universität Duisburg-Essen}\\
	  \IfLanguageName{english}{Faculty of Business Administration and Economics}{Fakultät für Wirtschaftswissensschaften}\\
	  \IfLanguageName{english}{Chair of Econometrics}{Lehrstuhl für Ökonometrie}\\
  \end{minipage}
	\begin{minipage}{0.4\textwidth}
	  \begin{flushright}
  	  \vspace{-0.5cm}
      \IfLanguageName{english}{\includegraphics*[width=5cm]{Includes/duelogo_en.png}}{\includegraphics*[width=5cm]{Includes/duelogo_de.png}}
	  \end{flushright}
	\end{minipage}
  \\
  \vspace{1.5cm}
  \begin{center}
  \huge{A Functional Approach to (Parallelised) Monte Carlo
Simulation}\\
  \vspace{.25cm}
  \Large{Advanced R for Econometricians}\\
  \vspace{0.5cm}
  \large{Final Project}\\
  \vspace{1cm}
  \large{
  \IfLanguageName{english}{Submitted to the Faculty of \\ Business Administration and Economics \\at the \\University of Duisburg-Essen}{Vorgelegt der \\Fakultät für Wirtschaftswissenschaften der \\ Universität Duisburg-Essen}\\}
  \vspace{0.75cm}
  \large{\IfLanguageName{english}{from:}{von:}}\\
  \vspace{0.5cm}
  Alexander Langnau, Öcal Kaptan, Sunyoung Ji\\
  \end{center}
  \vspace{4cm}

  \noindent\begin{minipage}[t]{0.5\textwidth}
  \IfLanguageName{english}{Matriculation Number:}{Matrikelnummer}
  \end{minipage}
  \begin{minipage}[t]{0.7\textwidth}
  \hspace{1cm}232907, 230914, 229979
  \end{minipage}

  \noindent\begin{minipage}[t]{0.5\textwidth}
  \IfLanguageName{english}{Study Path:}{Studienfach:}
  \end{minipage}
  \begin{minipage}[t]{0.7\textwidth}
  \hspace{1cm}M.Sc. Econometircs
  \end{minipage}

  \noindent\begin{minipage}[t]{0.5\textwidth}
  \IfLanguageName{english}{Reviewer:}{Erstgutachter:}
  \end{minipage}
  \begin{minipage}[t]{0.7\textwidth}
  \hspace{1cm}Prof.~Dr.~Christoph Hanck
  \end{minipage}

  \noindent\begin{minipage}[t]{0.5\textwidth}
  \IfLanguageName{english}{Secondary Reviewer:}{Zweitgutachter}
  \end{minipage}
  \begin{minipage}[t]{0.7\textwidth}
  \hspace{1cm}M.Sc. Martin C. Arnold, M.Sc. Jens Klenke
  \end{minipage}

  \noindent\begin{minipage}[t]{0.5\textwidth}
  Semester:
  \end{minipage}
  \begin{minipage}[t]{0.7\textwidth}
  \hspace{1cm}\IfLanguageName{english}{\nth{1} Semester}{1. Fachsemester}
  \end{minipage}

  \noindent\begin{minipage}[t]{0.5\textwidth}
  \IfLanguageName{english}{Graduation (est.):}{Vsl. Studienabschluss:}
  \end{minipage}
  \begin{minipage}[t]{0.7\textwidth}
  \hspace{1cm}Summer Term 2022
  \end{minipage}

  \noindent\begin{minipage}[t]{0.5\textwidth}
  \IfLanguageName{english}{Deadline:}{Abgabefrist:}
  \end{minipage}
  \begin{minipage}[t]{0.7\textwidth}
  \hspace{1cm}09. 09. 2022
  \end{minipage}

\end{titlepage}

% Ends the declared geometry for the titlepage
\restoregeometry


\pagenumbering{Roman} 
{
\hypersetup{linkcolor=black}
\setcounter{tocdepth}{3}
\tableofcontents
}
\newpage
\listoftables
\newpage
\listoffigures
\newpage
\pagenumbering{arabic} 
\hypertarget{introduction}{%
\section{Introduction}\label{introduction}}

Monte Carlo simulates complex probabilistic events using simple random
events, such as the tossing of a pair of dice to simulate the casino's
overall business model. In Monte Carlo computing, a pseudo-random number
generator is repeatedly called which returns a real number in {[}0,
1{]}, and the results are used to generate a distribution of samples
that is a fair representation of the target probability distribution
under study \autocite{Barbu_2022}. Monte Carlo Method is combined with
programming in modern research and contributes to various studies in
statistics, economics, and many other science fields. The paper make a
progress on developing a collection of different wrapper functions. The
main function provides a convenient interface for Monte Carlo
simulations and allows users to create a parameter grid and to iterate
homogenous function calls over the parameter grid. It also offers an
informative summary statistics including visualization with
ggplot-methods and an option to use a parallelization process by using
\texttt{furrr} package. The paper proceeds as follows. Chapter 2
describes preprocesses to establish the Monte Carlo simulation function.
The preprocesses includes functions to create grid and dataset with
random variables in user-defined distributions, along with functions
that provide summary statistics. Chanter 3 details the main Monte Carlo
simulation function which consists of functions in the chapter 2.
Chapter 4 presents examples with the main function. Finally, chapter 5
concludes. Each chapter contains a simple example to show that a
function is applicable to as many cases as possible. If there is a
restriction, It will be covered and discussed as well.

\hypertarget{preprocess-creating-helper-functions}{%
\section{Preprocess: Creating Helper
functions}\label{preprocess-creating-helper-functions}}

\hypertarget{function-for-creating-grid}{%
\subsection{Function for creating
grid}\label{function-for-creating-grid}}

\texttt{create\_grid} is one of the functions to improve performance of
the main Monte Carlo simulation function. That creates a hyper-parameter
grid with all permutations of the given parameters. Hyper-parameters are
the variables that are required to be introduced before implementing a
learning algorithm. It is typically unknown in advance about the
hyper-parameters that should be harmonized, their valid ranges and which
values in these ranges are most likely to yield a high
performance\autocite{Rana_2022}. Users can make their combination of
hyper-parameters, then apply it into MC simulation in the main function.

\begin{Shaded}
\begin{Highlighting}[]
\NormalTok{create\_grid }\OtherTok{\textless{}{-}} \ControlFlowTok{function}\NormalTok{(parameters, nrep)\{}
\NormalTok{  input }\OtherTok{\textless{}{-}}\NormalTok{ parameters}
\NormalTok{  storage }\OtherTok{\textless{}{-}} \FunctionTok{list}\NormalTok{()}
\NormalTok{  name\_vec }\OtherTok{\textless{}{-}} \FunctionTok{c}\NormalTok{()}
  
  \ControlFlowTok{for}\NormalTok{(i }\ControlFlowTok{in} \DecValTok{1}\SpecialCharTok{:}\FunctionTok{length}\NormalTok{(input))\{ }\CommentTok{\#1:3}
\NormalTok{    a }\OtherTok{\textless{}{-}} \FunctionTok{as.numeric}\NormalTok{(input[[i]][[}\DecValTok{2}\NormalTok{]])}
\NormalTok{    b }\OtherTok{\textless{}{-}} \FunctionTok{as.numeric}\NormalTok{(input[[i]][[}\DecValTok{3}\NormalTok{]])}
\NormalTok{    c }\OtherTok{\textless{}{-}} \FunctionTok{as.numeric}\NormalTok{(input[[i]][[}\DecValTok{4}\NormalTok{]])}
\NormalTok{    output }\OtherTok{\textless{}{-}} \FunctionTok{seq}\NormalTok{(}\AttributeTok{from=}\NormalTok{a, }\AttributeTok{to=}\NormalTok{b, }\AttributeTok{by=}\NormalTok{c)}
\NormalTok{    storage[[i]] }\OtherTok{\textless{}{-}}\NormalTok{  output}
\NormalTok{    name\_vec[i] }\OtherTok{\textless{}{-}}\NormalTok{ input[[i]][[}\DecValTok{1}\NormalTok{]]}
\NormalTok{  \}}
  
\NormalTok{  grid }\OtherTok{\textless{}{-}} \FunctionTok{expand\_grid}\NormalTok{(}\FunctionTok{unlist}\NormalTok{(storage[}\DecValTok{1}\NormalTok{])}
\NormalTok{                      , }\FunctionTok{unlist}\NormalTok{(storage[}\DecValTok{2}\NormalTok{])}
\NormalTok{                      , }\FunctionTok{unlist}\NormalTok{(storage[}\DecValTok{3}\NormalTok{])}
\NormalTok{                      , }\FunctionTok{unlist}\NormalTok{(storage[}\DecValTok{4}\NormalTok{])}
\NormalTok{                      , }\FunctionTok{unlist}\NormalTok{(storage[}\DecValTok{5}\NormalTok{])}
\NormalTok{                      , }\FunctionTok{c}\NormalTok{(}\DecValTok{1}\SpecialCharTok{:}\NormalTok{nrep))}
  
  \FunctionTok{names}\NormalTok{(grid) }\OtherTok{\textless{}{-}} \FunctionTok{c}\NormalTok{(name\_vec, }\StringTok{"rep"}\NormalTok{)}
  
  \FunctionTok{return}\NormalTok{(grid)}
\NormalTok{\}}
\end{Highlighting}
\end{Shaded}

Users have to input parameters as a list as following:

\begin{verbatim}
parameter_list <- list(c("variable name 1", from, to, by) 
                      ,c("variable name 2", from, to, by)
                      ,c("variable name 3", from, to, by)
                      ,c("variable name 4", from, to, by))
\end{verbatim}

\texttt{parameter\_list} works with a minimum of 1 and a maximum of 4
variables. The structure of arguments is similar to \texttt{seq()} in R:
A line of arguments is composed with variable name, the start and the
end of sequence, the steps. It would be fairly easy to adapt this helper
function for more parameters, but it is assumed that a grid with up to 4
parameters offers enough complexity for the simulation. The function
basically takes the information of the input parameter list and creates
a grid with \texttt{tidyr::expand\_grid()}. The structure of
\texttt{create\_grid} makes sure that the columns are located after the
corresponding variable and creates a different row for each number of
repetition(\texttt{nrep}).

\textbf{\texttt{create\_grid()} Example:}

\begin{Shaded}
\begin{Highlighting}[]
\CommentTok{\#four parameters }
\NormalTok{param\_list0 }\OtherTok{\textless{}{-}} \FunctionTok{list}\NormalTok{(}\FunctionTok{c}\NormalTok{(}\StringTok{"n"}\NormalTok{, }\DecValTok{10}\NormalTok{, }\DecValTok{20}\NormalTok{, }\DecValTok{10}\NormalTok{)}
\NormalTok{                    ,}\FunctionTok{c}\NormalTok{(}\StringTok{"mu"}\NormalTok{, }\DecValTok{0}\NormalTok{, }\FloatTok{0.5}\NormalTok{, }\FloatTok{0.25}\NormalTok{)}
\NormalTok{                    ,}\FunctionTok{c}\NormalTok{(}\StringTok{"sd"}\NormalTok{, }\DecValTok{0}\NormalTok{, }\FloatTok{0.3}\NormalTok{, }\FloatTok{0.1}\NormalTok{)}
\NormalTok{                    ,}\FunctionTok{c}\NormalTok{(}\StringTok{"gender"}\NormalTok{, }\DecValTok{0}\NormalTok{, }\DecValTok{1}\NormalTok{, }\DecValTok{1}\NormalTok{))}

\FunctionTok{head}\NormalTok{(}\FunctionTok{create\_grid}\NormalTok{(param\_list0, }\AttributeTok{nrep=}\DecValTok{3}\NormalTok{), }\AttributeTok{n=}\DecValTok{2}\NormalTok{)}
\end{Highlighting}
\end{Shaded}

\begin{verbatim}
## # A tibble: 2 x 5
##       n    mu    sd gender   rep
##   <dbl> <dbl> <dbl>  <dbl> <int>
## 1    10     0     0      0     1
## 2    10     0     0      0     2
\end{verbatim}

\hypertarget{data-generation-function}{%
\subsection{Data generation function}\label{data-generation-function}}

\texttt{data\_generation()} takes \texttt{grid} and \texttt{simulation}
as inputs. Users can define a probability distribution of data by
entering the name of R packages into the function, such as normal
distribution(\texttt{rnorm}) and uniform distribution(\texttt{runif}).

The draw of data points for each row of the parameter grid gets stored
as a seperate element in a list.

\texttt{data\_generation()} chooses a mapping function itself based on
the number of parameters. Mapping functions that belong to
\texttt{purrr} package are below:

\begin{longtable}[]{@{}lc@{}}
\toprule()
Map & Number of parameters \\
\midrule()
\endhead
map() & \(\leq1\) \\
map2() & \(\leq2\) \\
pmap & otherwise \\
\bottomrule()
\end{longtable}

\begin{Shaded}
\begin{Highlighting}[]
\NormalTok{data\_generation }\OtherTok{\textless{}{-}} \ControlFlowTok{function}\NormalTok{(simulation, grid)\{ }

  
  \ControlFlowTok{if}\NormalTok{(}\FunctionTok{ncol}\NormalTok{(grid)}\SpecialCharTok{==}\DecValTok{2}\NormalTok{)\{}
\NormalTok{    var1 }\OtherTok{\textless{}{-}} \FunctionTok{c}\NormalTok{(}\FunctionTok{unlist}\NormalTok{(grid[,}\DecValTok{1}\NormalTok{]))}
\NormalTok{    data }\OtherTok{\textless{}{-}} \FunctionTok{map}\NormalTok{(var1, simulation) }
\NormalTok{  \}}
  
  \ControlFlowTok{if}\NormalTok{(}\FunctionTok{ncol}\NormalTok{(grid)}\SpecialCharTok{==}\DecValTok{3}\NormalTok{)\{}
\NormalTok{    var1 }\OtherTok{\textless{}{-}} \FunctionTok{c}\NormalTok{(}\FunctionTok{unlist}\NormalTok{(grid[,}\DecValTok{1}\NormalTok{]))}
\NormalTok{    var2 }\OtherTok{\textless{}{-}} \FunctionTok{c}\NormalTok{(}\FunctionTok{unlist}\NormalTok{(grid[,}\DecValTok{2}\NormalTok{]))}
\NormalTok{    data }\OtherTok{\textless{}{-}} \FunctionTok{map2}\NormalTok{(var1, var2, simulation)}
\NormalTok{  \} }
  
  \ControlFlowTok{if}\NormalTok{(}\FunctionTok{ncol}\NormalTok{(grid)}\SpecialCharTok{==}\DecValTok{4}\NormalTok{)\{ }
\NormalTok{    var1 }\OtherTok{\textless{}{-}} \FunctionTok{c}\NormalTok{(}\FunctionTok{unlist}\NormalTok{(grid[,}\DecValTok{1}\NormalTok{]))}
\NormalTok{    var2 }\OtherTok{\textless{}{-}} \FunctionTok{c}\NormalTok{(}\FunctionTok{unlist}\NormalTok{(grid[,}\DecValTok{2}\NormalTok{]))}
\NormalTok{    var3 }\OtherTok{\textless{}{-}} \FunctionTok{c}\NormalTok{(}\FunctionTok{unlist}\NormalTok{(grid[,}\DecValTok{3}\NormalTok{]))}
\NormalTok{    list1 }\OtherTok{\textless{}{-}} \FunctionTok{list}\NormalTok{(var1,var2,var3)}
\NormalTok{    data }\OtherTok{\textless{}{-}} \FunctionTok{pmap}\NormalTok{(list1, }\AttributeTok{.f=}\NormalTok{simulation)}
\NormalTok{  \} }
  
  \FunctionTok{return}\NormalTok{(data)}
\NormalTok{\}}
\end{Highlighting}
\end{Shaded}

Monte Carlo simulations can become quickly very demanding in terms of
computing time. Parallelisation is a way of BLABLABLA

(\url{https://nceas.github.io/oss-lessons/parallel-computing-in-r/parallel-computing-in-r.html})

In case the user is looking to improve performance, a way of choosing a
parallelised processing function of the \texttt{furrr}-package is
offered. At this point we waive running an example with the parallelised
version of our data generating function, since the output wouldn't look
different compared to the previous example. At a later point in the
paper we´ll showcase the difference in computation time for the
different functions.

\begin{Shaded}
\begin{Highlighting}[]
\NormalTok{data\_generation\_parallelised }\OtherTok{\textless{}{-}} \ControlFlowTok{function}\NormalTok{(simulation, grid)\{ }\CommentTok{\#this is for use inside the function}
  
  \ControlFlowTok{if}\NormalTok{(}\FunctionTok{ncol}\NormalTok{(grid)}\SpecialCharTok{==}\DecValTok{2}\NormalTok{)\{}
\NormalTok{    var1 }\OtherTok{\textless{}{-}} \FunctionTok{c}\NormalTok{(}\FunctionTok{unlist}\NormalTok{(grid[,}\DecValTok{1}\NormalTok{]))}
\NormalTok{    data }\OtherTok{\textless{}{-}} \FunctionTok{future\_map}\NormalTok{(var1, simulation,}\AttributeTok{.options =} \FunctionTok{furrr\_options}\NormalTok{(}\AttributeTok{seed =} \ConstantTok{TRUE}\NormalTok{))}
    
\NormalTok{  \}}
  
  \ControlFlowTok{if}\NormalTok{(}\FunctionTok{ncol}\NormalTok{(grid)}\SpecialCharTok{==}\DecValTok{3}\NormalTok{)\{}
\NormalTok{    var1 }\OtherTok{\textless{}{-}} \FunctionTok{c}\NormalTok{(}\FunctionTok{unlist}\NormalTok{(grid[,}\DecValTok{1}\NormalTok{]))}
\NormalTok{    var2 }\OtherTok{\textless{}{-}} \FunctionTok{c}\NormalTok{(}\FunctionTok{unlist}\NormalTok{(grid[,}\DecValTok{2}\NormalTok{]))}
\NormalTok{    data }\OtherTok{\textless{}{-}} \FunctionTok{future\_map2}\NormalTok{(var1, var2, simulation,}\AttributeTok{.options =} \FunctionTok{furrr\_options}\NormalTok{(}\AttributeTok{seed =} \ConstantTok{TRUE}\NormalTok{))}
\NormalTok{  \} }
  
  \ControlFlowTok{if}\NormalTok{(}\FunctionTok{ncol}\NormalTok{(grid)}\SpecialCharTok{==}\DecValTok{4}\NormalTok{)\{ }\CommentTok{\#need to implement more than 3?!}
\NormalTok{    var1 }\OtherTok{\textless{}{-}} \FunctionTok{c}\NormalTok{(}\FunctionTok{unlist}\NormalTok{(grid[,}\DecValTok{1}\NormalTok{]))}
\NormalTok{    var2 }\OtherTok{\textless{}{-}} \FunctionTok{c}\NormalTok{(}\FunctionTok{unlist}\NormalTok{(grid[,}\DecValTok{2}\NormalTok{]))}
\NormalTok{    var3 }\OtherTok{\textless{}{-}} \FunctionTok{c}\NormalTok{(}\FunctionTok{unlist}\NormalTok{(grid[,}\DecValTok{3}\NormalTok{]))}
\NormalTok{    list1 }\OtherTok{\textless{}{-}} \FunctionTok{list}\NormalTok{(var1,var2,var3)}
\NormalTok{    data }\OtherTok{\textless{}{-}} \FunctionTok{future\_pmap}\NormalTok{(list1, }\AttributeTok{.f=}\NormalTok{simulation,}\AttributeTok{.options =} \FunctionTok{furrr\_options}\NormalTok{(}\AttributeTok{seed =} \ConstantTok{TRUE}\NormalTok{))}
\NormalTok{  \} }
  
  \FunctionTok{return}\NormalTok{(data)}
\NormalTok{\}}
\end{Highlighting}
\end{Shaded}

Following, we´d like to demonstrate the unparallelised function with a
simple example using the normal distribution (\texttt{rpois()}) as the
underlying data generating process:

\textbf{\texttt{data\_generation()} Example:}

\begin{Shaded}
\begin{Highlighting}[]
\CommentTok{\#param\_list1 \textless{}{-} list(c("n", 10, 20, 10))}

\NormalTok{param\_list2 }\OtherTok{\textless{}{-}} \FunctionTok{list}\NormalTok{(}\FunctionTok{c}\NormalTok{(}\StringTok{"n"}\NormalTok{, }\DecValTok{10}\NormalTok{, }\DecValTok{20}\NormalTok{, }\DecValTok{10}\NormalTok{)}
\NormalTok{                  ,}\FunctionTok{c}\NormalTok{(}\StringTok{"lambda"}\NormalTok{, }\FloatTok{0.5}\NormalTok{, }\DecValTok{1}\NormalTok{, }\FloatTok{0.5}\NormalTok{))}
\CommentTok{\#create\_grid(param\_list1, nrep=10)}
\CommentTok{\#create\_grid(param\_list1, nrep=1)}

\StringTok{\textquotesingle{}grid1 \textless{}{-} create\_grid(param\_list1, nrep=3)}
\StringTok{tail(data\_generation(simulation=rnorm, grid=grid1),1)\textquotesingle{}}
\end{Highlighting}
\end{Shaded}

\begin{verbatim}
## [1] "grid1 <- create_grid(param_list1, nrep=3)\ntail(data_generation(simulation=rnorm, grid=grid1),1)"
\end{verbatim}

\begin{Shaded}
\begin{Highlighting}[]
\NormalTok{grid2 }\OtherTok{\textless{}{-}} \FunctionTok{create\_grid}\NormalTok{(param\_list2, }\AttributeTok{nrep=}\DecValTok{1}\NormalTok{)}
\NormalTok{sim1 }\OtherTok{\textless{}{-}} \FunctionTok{data\_generation}\NormalTok{(}\AttributeTok{simulation=}\NormalTok{rpois, }\AttributeTok{grid=}\NormalTok{grid2)}

\NormalTok{grid2}
\end{Highlighting}
\end{Shaded}

\begin{verbatim}
## # A tibble: 4 x 3
##       n lambda   rep
##   <dbl>  <dbl> <int>
## 1    10    0.5     1
## 2    10    1       1
## 3    20    0.5     1
## 4    20    1       1
\end{verbatim}

\begin{Shaded}
\begin{Highlighting}[]
\FunctionTok{str}\NormalTok{(sim1)}
\end{Highlighting}
\end{Shaded}

\begin{verbatim}
## List of 4
##  $ n1: int [1:10] 0 1 0 1 2 0 0 1 0 0
##  $ n2: int [1:10] 3 1 1 1 0 2 0 0 0 3
##  $ n3: int [1:20] 1 1 1 3 1 1 0 0 0 0 ...
##  $ n4: int [1:20] 0 1 1 1 0 0 0 1 0 2 ...
\end{verbatim}

\begin{Shaded}
\begin{Highlighting}[]
\NormalTok{sim1}
\end{Highlighting}
\end{Shaded}

\begin{verbatim}
## $n1
##  [1] 0 1 0 1 2 0 0 1 0 0
## 
## $n2
##  [1] 3 1 1 1 0 2 0 0 0 3
## 
## $n3
##  [1] 1 1 1 3 1 1 0 0 0 0 2 1 1 1 0 0 1 0 0 0
## 
## $n4
##  [1] 0 1 1 1 0 0 0 1 0 2 0 1 2 0 1 0 0 2 2 1
\end{verbatim}

We see a list containg one variable for each row of the parameter grid,
where all the generated data points are stored. For example, the last
row (stored under \texttt{sim1\$n4}) specified n = 10 draws from the
normal distribution with \(\lambda = 1\). Since we set
\texttt{nrep\ =\ 1} in order to save space, the draw only happened once.

The format list offers alot of flexibility, but is not very overseeable.
At this point we can use the raw data to run summary statistics on,
which we´ll do in the next passage.

\begin{Shaded}
\begin{Highlighting}[]
\CommentTok{\# Application to Uniform distribution}
\NormalTok{param\_list\_runif }\OtherTok{\textless{}{-}} \FunctionTok{list}\NormalTok{(}\FunctionTok{c}\NormalTok{(}\StringTok{"n"}\NormalTok{, }\DecValTok{10}\NormalTok{, }\DecValTok{30}\NormalTok{, }\DecValTok{10}\NormalTok{)}
\NormalTok{                         ,}\FunctionTok{c}\NormalTok{(}\StringTok{"min"}\NormalTok{, }\DecValTok{0}\NormalTok{, }\DecValTok{0}\NormalTok{, }\DecValTok{0}\NormalTok{)}
\NormalTok{                         ,}\FunctionTok{c}\NormalTok{(}\StringTok{"max"}\NormalTok{, }\DecValTok{1}\NormalTok{, }\DecValTok{1}\NormalTok{, }\DecValTok{0}\NormalTok{))}


\NormalTok{grid\_unif }\OtherTok{\textless{}{-}} \FunctionTok{create\_grid}\NormalTok{(param\_list\_runif, }\AttributeTok{nrep=}\DecValTok{3}\NormalTok{)}
\FunctionTok{tail}\NormalTok{(}\FunctionTok{data\_generation}\NormalTok{(}\AttributeTok{simulation=}\NormalTok{runif, }\AttributeTok{grid=}\NormalTok{grid\_unif),}\DecValTok{1}\NormalTok{)}
\end{Highlighting}
\end{Shaded}

\begin{verbatim}
## $n9
##  [1] 0.48204261 0.25296493 0.21625479 0.67437639 0.04766363 0.70085309
##  [7] 0.35188864 0.40894400 0.82095132 0.91885735 0.28252833 0.96110479
## [13] 0.72839443 0.68637508 0.05284394 0.39522013 0.47784538 0.56025326
## [19] 0.69826159 0.91568354 0.61835123 0.42842151 0.54208037 0.05847849
## [25] 0.26085686 0.39715195 0.19774474 0.83192756 0.15288722 0.80341854
\end{verbatim}

\begin{Shaded}
\begin{Highlighting}[]
\CommentTok{\# Application to Poisson distribution}

\NormalTok{param\_list\_rpois }\OtherTok{\textless{}{-}} \FunctionTok{list}\NormalTok{(}\FunctionTok{c}\NormalTok{(}\StringTok{"n"}\NormalTok{, }\DecValTok{10}\NormalTok{, }\DecValTok{30}\NormalTok{, }\DecValTok{10}\NormalTok{)}
\NormalTok{                         , }\FunctionTok{c}\NormalTok{(}\StringTok{"lambda"}\NormalTok{, }\DecValTok{0}\NormalTok{, }\DecValTok{10}\NormalTok{, }\DecValTok{1}\NormalTok{))}

\NormalTok{grid\_pois }\OtherTok{\textless{}{-}} \FunctionTok{create\_grid}\NormalTok{(param\_list\_rpois, }\AttributeTok{nrep=}\DecValTok{3}\NormalTok{)}
\FunctionTok{tail}\NormalTok{(grid\_pois,}\DecValTok{2}\NormalTok{) }\CommentTok{\# nrow(grid\_pois) = 99}
\end{Highlighting}
\end{Shaded}

\begin{verbatim}
## # A tibble: 2 x 3
##       n lambda   rep
##   <dbl>  <dbl> <int>
## 1    30     10     2
## 2    30     10     3
\end{verbatim}

\begin{Shaded}
\begin{Highlighting}[]
\FunctionTok{tail}\NormalTok{(}\FunctionTok{data\_generation}\NormalTok{(}\AttributeTok{simulation=}\NormalTok{rpois, }\AttributeTok{grid=}\NormalTok{grid\_pois),}\DecValTok{1}\NormalTok{)}
\end{Highlighting}
\end{Shaded}

\begin{verbatim}
## $n99
##  [1] 11  8 11  6  5  8 11  9 13 12  9  8 14 16  8 20  9  8 10  5 13 14  4  8 10
## [26] 10 10  7  9  7
\end{verbatim}

\hypertarget{summary-function}{%
\subsection{Summary function}\label{summary-function}}

Using the tools we showed before the user is able to generate the
\emph{raw} data from an underlying distribution of his choice. The next
step is to introduce a way of applying a defined summary statistics onto
that data, which we realised using the function called
\texttt{summary\_function()}.

This function basically just applys the user defined summary function
(under the input \texttt{sum\_fun}) onto the raw data using a
\texttt{sapply()}-loop. The results gets stored in a nrow(grid ) x 1
dimensional matrix, which will be combined with the parameter grid in
the next step, in order to correctly allocate each result to the related
set of parameters.

\begin{Shaded}
\begin{Highlighting}[]
\CommentTok{\#summary function for one input}
\NormalTok{summary\_function }\OtherTok{\textless{}{-}} \ControlFlowTok{function}\NormalTok{(sum\_fun, data\_input)\{}
  
\NormalTok{  count }\OtherTok{\textless{}{-}} \FunctionTok{length}\NormalTok{(data\_input)}
\NormalTok{  summary\_matrix }\OtherTok{\textless{}{-}} \FunctionTok{matrix}\NormalTok{(}\AttributeTok{nrow=}\NormalTok{count, }\AttributeTok{ncol=}\DecValTok{1}\NormalTok{)}
  
  \ControlFlowTok{for}\NormalTok{(i }\ControlFlowTok{in} \DecValTok{1}\SpecialCharTok{:}\NormalTok{count)\{}
\NormalTok{    input }\OtherTok{\textless{}{-}} \FunctionTok{list}\NormalTok{(data\_input[[i]])}
\NormalTok{    output }\OtherTok{\textless{}{-}} \FunctionTok{sapply}\NormalTok{(sum\_fun, do.call, input)}
\NormalTok{    summary\_matrix[i] }\OtherTok{\textless{}{-}}\NormalTok{ output}
\NormalTok{  \}}
  \CommentTok{\#output \textless{}{-} as.data.frame(summary\_matrix)}
  \CommentTok{\#names(output) \textless{}{-} sum\_fun}
  \FunctionTok{colnames}\NormalTok{(summary\_matrix) }\OtherTok{\textless{}{-}}\NormalTok{ sum\_fun}
  \FunctionTok{return}\NormalTok{(summary\_matrix)}
\NormalTok{\}}
\end{Highlighting}
\end{Shaded}

The example to demonstrate this function uses rnorm()-function as
underlying DGB, where we specified a parameter grid over three
parameters (n, \(\mu\) and the standard deviation).

\textbf{\texttt{summary\_function} Example:}

\begin{Shaded}
\begin{Highlighting}[]
\NormalTok{param\_list3 }\OtherTok{\textless{}{-}} \FunctionTok{list}\NormalTok{(}\FunctionTok{c}\NormalTok{(}\StringTok{"n"}\NormalTok{, }\DecValTok{10}\NormalTok{, }\DecValTok{20}\NormalTok{, }\DecValTok{10}\NormalTok{)}
\NormalTok{                    ,}\FunctionTok{c}\NormalTok{(}\StringTok{"mu"}\NormalTok{, }\DecValTok{1}\NormalTok{, }\DecValTok{2}\NormalTok{, }\FloatTok{0.25}\NormalTok{)}
\NormalTok{                    ,}\FunctionTok{c}\NormalTok{(}\StringTok{"sd"}\NormalTok{, }\FloatTok{0.5}\NormalTok{, }\DecValTok{1}\NormalTok{, }\FloatTok{0.1}\NormalTok{))}

\NormalTok{grid\_test }\OtherTok{\textless{}{-}} \FunctionTok{create\_grid}\NormalTok{(param\_list3, }\AttributeTok{nrep=}\DecValTok{3}\NormalTok{)}
\NormalTok{test\_data }\OtherTok{\textless{}{-}} \FunctionTok{data\_generation}\NormalTok{(}\AttributeTok{simulation=}\NormalTok{rnorm, }\AttributeTok{grid=}\NormalTok{grid\_test)}
\NormalTok{summary\_data }\OtherTok{\textless{}{-}} \FunctionTok{summary\_function}\NormalTok{(}\AttributeTok{sum\_fun=}\FunctionTok{list}\NormalTok{(}\StringTok{"mean"}\NormalTok{), }\AttributeTok{data\_input=}\NormalTok{test\_data)}
\FunctionTok{head}\NormalTok{(summary\_data)}
\end{Highlighting}
\end{Shaded}

\begin{verbatim}
##           mean
## [1,] 1.0165396
## [2,] 0.8156316
## [3,] 1.0361343
## [4,] 1.0129142
## [5,] 1.0714922
## [6,] 1.1017460
\end{verbatim}

\hypertarget{summary-array-funcation}{%
\subsection{Summary array funcation}\label{summary-array-funcation}}

Even tough we specified a fairly small parameter grid in the example
above, our simulation consisted retuned 180 summarised data points for
the specified simulation. In the main\_function() the results from the
previous step get merged with the parameter grid into one data frame.
This way of storing the data allows the user to apply further data
wrangling processes, but is not suitable for printing the output in a
tidy and clear way. A multt-dimensional array is better suited for this
case.

The function \texttt{create\_array\_function()} takes all relevant data
from the steps before (parameter grid and the results of the Monte Carlo
simulation) and transforms it into an array with the correct dimensions.

\begin{Shaded}
\begin{Highlighting}[]
\NormalTok{create\_array\_function }\OtherTok{\textless{}{-}} \ControlFlowTok{function}\NormalTok{(comb, parameters, nrep)\{}
\NormalTok{  storage }\OtherTok{\textless{}{-}} \FunctionTok{list}\NormalTok{()}
\NormalTok{  name\_vec }\OtherTok{\textless{}{-}} \FunctionTok{c}\NormalTok{()}
  
  \ControlFlowTok{for}\NormalTok{(i }\ControlFlowTok{in} \DecValTok{1}\SpecialCharTok{:}\FunctionTok{length}\NormalTok{(parameters))\{ }
    \CommentTok{\#this creates the sequences of parameters}
\NormalTok{    a }\OtherTok{\textless{}{-}} \FunctionTok{as.numeric}\NormalTok{(parameters[[i]][[}\DecValTok{2}\NormalTok{]])}
\NormalTok{    b }\OtherTok{\textless{}{-}} \FunctionTok{as.numeric}\NormalTok{(parameters[[i]][[}\DecValTok{3}\NormalTok{]])}
\NormalTok{    c }\OtherTok{\textless{}{-}} \FunctionTok{as.numeric}\NormalTok{(parameters[[i]][[}\DecValTok{4}\NormalTok{]])}
\NormalTok{    output }\OtherTok{\textless{}{-}} \FunctionTok{seq}\NormalTok{(}\AttributeTok{from=}\NormalTok{a, }\AttributeTok{to=}\NormalTok{b, }\AttributeTok{by=}\NormalTok{c)}
\NormalTok{    storage[[i]] }\OtherTok{\textless{}{-}}\NormalTok{  output}
\NormalTok{    name\_vec[i] }\OtherTok{\textless{}{-}}\NormalTok{ parameters[[i]][[}\DecValTok{1}\NormalTok{]] }
    \CommentTok{\#this just stores the names of the variables}
\NormalTok{  \}}
  
  
\NormalTok{  matrix.numeration }\OtherTok{\textless{}{-}}  \FunctionTok{paste}\NormalTok{(}\StringTok{"rep"}\NormalTok{,}\StringTok{"="}\NormalTok{, }\DecValTok{1}\SpecialCharTok{:}\NormalTok{nrep, }\AttributeTok{sep =} \StringTok{""}\NormalTok{)}
  
  \ControlFlowTok{if}\NormalTok{(}\FunctionTok{length}\NormalTok{(parameters)}\SpecialCharTok{==}\DecValTok{1}\NormalTok{)\{}
\NormalTok{    comb\_ordered }\OtherTok{\textless{}{-}}\NormalTok{  comb }\SpecialCharTok{\%\textgreater{}\%} \FunctionTok{arrange}\NormalTok{(comb[,}\DecValTok{2}\NormalTok{])}
\NormalTok{    seq1 }\OtherTok{\textless{}{-}} \FunctionTok{c}\NormalTok{(}\FunctionTok{unlist}\NormalTok{(storage[}\DecValTok{1}\NormalTok{]))}
    
\NormalTok{    row.names }\OtherTok{\textless{}{-}} \FunctionTok{paste}\NormalTok{(name\_vec[}\DecValTok{1}\NormalTok{],}\StringTok{"="}\NormalTok{,seq1, }\AttributeTok{sep =} \StringTok{""}\NormalTok{)}
    
\NormalTok{    dimension\_array }\OtherTok{\textless{}{-}} \FunctionTok{c}\NormalTok{(}\FunctionTok{length}\NormalTok{(seq1), nrep)}
\NormalTok{    dim\_names\_list }\OtherTok{\textless{}{-}} \FunctionTok{list}\NormalTok{(row.names, matrix.numeration)}
\NormalTok{  \}}
  
  \ControlFlowTok{if}\NormalTok{(}\FunctionTok{length}\NormalTok{(parameters)}\SpecialCharTok{==}\DecValTok{2}\NormalTok{)\{}
\NormalTok{    comb\_ordered }\OtherTok{\textless{}{-}}\NormalTok{  comb }\SpecialCharTok{\%\textgreater{}\%} \FunctionTok{arrange}\NormalTok{(comb[,}\DecValTok{2}\NormalTok{])  }\SpecialCharTok{\%\textgreater{}\%} \FunctionTok{arrange}\NormalTok{(comb[,}\DecValTok{3}\NormalTok{])}
\NormalTok{    seq1 }\OtherTok{\textless{}{-}} \FunctionTok{c}\NormalTok{(}\FunctionTok{unlist}\NormalTok{(storage[}\DecValTok{1}\NormalTok{]))}
\NormalTok{    seq2 }\OtherTok{\textless{}{-}} \FunctionTok{c}\NormalTok{(}\FunctionTok{unlist}\NormalTok{(storage[}\DecValTok{2}\NormalTok{]))}
    
\NormalTok{    row.names }\OtherTok{\textless{}{-}} \FunctionTok{paste}\NormalTok{(name\_vec[}\DecValTok{1}\NormalTok{],}\StringTok{"="}\NormalTok{,seq1, }\AttributeTok{sep =} \StringTok{""}\NormalTok{)}
\NormalTok{    column.names }\OtherTok{\textless{}{-}}  \FunctionTok{paste}\NormalTok{(name\_vec[}\DecValTok{2}\NormalTok{],}\StringTok{"="}\NormalTok{,seq2, }\AttributeTok{sep =} \StringTok{""}\NormalTok{)}
    
\NormalTok{    dimension\_array }\OtherTok{\textless{}{-}} \FunctionTok{c}\NormalTok{(}\FunctionTok{length}\NormalTok{(seq1), }\FunctionTok{length}\NormalTok{(seq2), nrep)}
\NormalTok{    dim\_names\_list }\OtherTok{\textless{}{-}} \FunctionTok{list}\NormalTok{(row.names, column.names, matrix.numeration)}
\NormalTok{  \}}
  
  \ControlFlowTok{if}\NormalTok{(}\FunctionTok{length}\NormalTok{(parameters)}\SpecialCharTok{==}\DecValTok{3}\NormalTok{)\{}
\NormalTok{    comb\_ordered }\OtherTok{\textless{}{-}}\NormalTok{  comb }\SpecialCharTok{\%\textgreater{}\%} \FunctionTok{arrange}\NormalTok{(comb[,}\DecValTok{2}\NormalTok{])  }\SpecialCharTok{\%\textgreater{}\%} 
      \FunctionTok{arrange}\NormalTok{(comb[,}\DecValTok{3}\NormalTok{]) }\SpecialCharTok{\%\textgreater{}\%} \FunctionTok{arrange}\NormalTok{(comb[,}\DecValTok{4}\NormalTok{]) }
\NormalTok{    seq1 }\OtherTok{\textless{}{-}} \FunctionTok{c}\NormalTok{(}\FunctionTok{unlist}\NormalTok{(storage[}\DecValTok{1}\NormalTok{]))}
\NormalTok{    seq2 }\OtherTok{\textless{}{-}} \FunctionTok{c}\NormalTok{(}\FunctionTok{unlist}\NormalTok{(storage[}\DecValTok{2}\NormalTok{]))}
\NormalTok{    seq3 }\OtherTok{\textless{}{-}} \FunctionTok{c}\NormalTok{(}\FunctionTok{unlist}\NormalTok{(storage[}\DecValTok{3}\NormalTok{]))}
    
\NormalTok{    row.names }\OtherTok{\textless{}{-}} \FunctionTok{paste}\NormalTok{(name\_vec[}\DecValTok{1}\NormalTok{],}\StringTok{"="}\NormalTok{,seq1, }\AttributeTok{sep =} \StringTok{""}\NormalTok{)}
\NormalTok{    column.names }\OtherTok{\textless{}{-}}  \FunctionTok{paste}\NormalTok{(name\_vec[}\DecValTok{2}\NormalTok{],}\StringTok{"="}\NormalTok{,seq2, }\AttributeTok{sep =} \StringTok{""}\NormalTok{)}
\NormalTok{    matrix.names1 }\OtherTok{\textless{}{-}}  \FunctionTok{paste}\NormalTok{(name\_vec[}\DecValTok{3}\NormalTok{],}\StringTok{"="}\NormalTok{,seq3, }\AttributeTok{sep =} \StringTok{""}\NormalTok{)}
    
\NormalTok{    dimension\_array }\OtherTok{\textless{}{-}} \FunctionTok{c}\NormalTok{(}\FunctionTok{length}\NormalTok{(seq1), }\FunctionTok{length}\NormalTok{(seq2), }\FunctionTok{length}\NormalTok{(seq3), nrep)}
\NormalTok{    dim\_names\_list }\OtherTok{\textless{}{-}} \FunctionTok{list}\NormalTok{(row.names, column.names, }
\NormalTok{                           matrix.names1, matrix.numeration)}
    
\NormalTok{  \}}
  
  
\NormalTok{  array1 }\OtherTok{\textless{}{-}} \FunctionTok{array}\NormalTok{(comb\_ordered[,}\FunctionTok{ncol}\NormalTok{(comb)] }
                  \CommentTok{\#change to automatically adjust dim}
\NormalTok{                  , }\AttributeTok{dim =}\NormalTok{ dimension\_array}
\NormalTok{                  , dim\_names\_list)}
  \FunctionTok{return}\NormalTok{(array1)}
\NormalTok{\}}
\end{Highlighting}
\end{Shaded}

In order to test this function, we need to set up an altered version of
the main\_function(), that is introduced in the next passage.

\textbf{\texttt{create\_array\_function} Example:}

\begin{Shaded}
\begin{Highlighting}[]
\CommentTok{\# PREP }\AlertTok{TEST}\CommentTok{ \textasciigrave{}create\_array\_function\textasciigrave{}}
\NormalTok{main\_function\_array\_test }\OtherTok{\textless{}{-}}  \ControlFlowTok{function}\NormalTok{(parameters }\CommentTok{\#list of parameters}
\NormalTok{                                      , nrep }\CommentTok{\#number of repetitions}
\NormalTok{                                      , simulation }\CommentTok{\#data genereation}
\NormalTok{                                      , sum\_fun)\{ }\CommentTok{\#summary statistics}
  
\NormalTok{  grid }\OtherTok{\textless{}{-}} \FunctionTok{create\_grid}\NormalTok{(parameters, nrep) }\CommentTok{\#Step 1: create grid}
\NormalTok{  raw\_data }\OtherTok{\textless{}{-}} \FunctionTok{data\_generation}\NormalTok{(simulation, grid) }\CommentTok{\#Step 2: simlate data}
\NormalTok{  summary }\OtherTok{\textless{}{-}} \FunctionTok{summary\_function}\NormalTok{(sum\_fun, }\AttributeTok{data\_input=}\NormalTok{raw\_data) }\CommentTok{\#Step 3: Summary statistics}
\NormalTok{  comb }\OtherTok{\textless{}{-}} \FunctionTok{cbind}\NormalTok{(grid, summary) }\CommentTok{\#Step 4: Combine resuluts with parameters}
\NormalTok{  array\_1 }\OtherTok{\textless{}{-}} \FunctionTok{create\_array\_function}\NormalTok{(comb, parameters, nrep) }\CommentTok{\#Step 5: Create array}
  
  \FunctionTok{return}\NormalTok{(comb)}
\NormalTok{\}}

\NormalTok{param\_list3x }\OtherTok{\textless{}{-}} \FunctionTok{list}\NormalTok{(}\FunctionTok{c}\NormalTok{(}\StringTok{"n"}\NormalTok{, }\DecValTok{10}\NormalTok{, }\DecValTok{20}\NormalTok{, }\DecValTok{10}\NormalTok{)}
\NormalTok{                     ,}\FunctionTok{c}\NormalTok{(}\StringTok{"mu"}\NormalTok{, }\DecValTok{0}\NormalTok{, }\DecValTok{5}\NormalTok{, }\DecValTok{1}\NormalTok{)}
\NormalTok{                     ,}\FunctionTok{c}\NormalTok{(}\StringTok{"sd"}\NormalTok{, }\DecValTok{0}\NormalTok{, }\DecValTok{3}\NormalTok{, }\DecValTok{1}\NormalTok{))}

\NormalTok{comb1 }\OtherTok{\textless{}{-}} \FunctionTok{main\_function\_array\_test}\NormalTok{(}\AttributeTok{parameters=}\NormalTok{param\_list3x}
\NormalTok{                                  , }\AttributeTok{nrep =} \DecValTok{3}
\NormalTok{                                  , }\AttributeTok{simulation =}\NormalTok{ rnorm}
\NormalTok{                                  , }\AttributeTok{sum\_fun=}\StringTok{"mean"}\NormalTok{)}

\FunctionTok{tail}\NormalTok{(comb1,}\DecValTok{2}\NormalTok{)}
\end{Highlighting}
\end{Shaded}

\begin{verbatim}
##      n mu sd rep     mean
## 143 20  5  3   2 6.304353
## 144 20  5  3   3 4.154558
\end{verbatim}

\begin{Shaded}
\begin{Highlighting}[]
\FunctionTok{create\_array\_function}\NormalTok{(}\AttributeTok{comb=}\NormalTok{comb1, }\AttributeTok{parameters=}\NormalTok{param\_list3x, }\AttributeTok{nrep=}\DecValTok{3}\NormalTok{)}
\end{Highlighting}
\end{Shaded}

\begin{verbatim}
## , , sd=0, rep=1
## 
##      mu=0 mu=1 mu=2 mu=3 mu=4 mu=5
## n=10    0    1    2    3    4    5
## n=20    0    1    2    3    4    5
## 
## , , sd=1, rep=1
## 
##            mu=0     mu=1     mu=2     mu=3     mu=4     mu=5
## n=10  0.3592213 1.856067 1.597645 2.631208 3.822198 5.589992
## n=20 -0.2024437 1.095379 2.207984 2.853540 3.701126 5.105401
## 
## , , sd=2, rep=1
## 
##            mu=0     mu=1     mu=2     mu=3     mu=4     mu=5
## n=10 -0.8768202 1.201310 1.904537 2.897435 3.509705 4.542338
## n=20 -0.6888252 1.395944 2.148185 2.966625 4.063234 5.059694
## 
## , , sd=3, rep=1
## 
##            mu=0     mu=1     mu=2     mu=3     mu=4     mu=5
## n=10 -0.3415539 1.284343 2.100330 2.699081 3.838544 3.959394
## n=20  0.7880765 1.004376 2.048212 2.978085 3.923330 4.814248
## 
## , , sd=0, rep=2
## 
##      mu=0 mu=1 mu=2 mu=3 mu=4 mu=5
## n=10    0    1    2    3    4    5
## n=20    0    1    2    3    4    5
## 
## , , sd=1, rep=2
## 
##            mu=0      mu=1     mu=2     mu=3     mu=4     mu=5
## n=10 -0.2452999 1.0835364 2.283412 3.593855 4.122586 4.401167
## n=20  0.4255240 0.8355293 1.804812 3.007484 3.752047 5.015939
## 
## , , sd=2, rep=2
## 
##             mu=0      mu=1     mu=2     mu=3     mu=4     mu=5
## n=10 -0.05139362 0.2426633 1.729941 1.968998 4.163818 4.394397
## n=20 -0.47500354 0.9687178 1.951855 2.666098 4.016383 4.273766
## 
## , , sd=3, rep=2
## 
##           mu=0      mu=1     mu=2     mu=3     mu=4     mu=5
## n=10 1.3917130 1.3173227 1.810426 3.500988 4.481365 7.218648
## n=20 0.5305719 0.5736219 1.831536 3.345939 4.261094 6.304353
## 
## , , sd=0, rep=3
## 
##      mu=0 mu=1 mu=2 mu=3 mu=4 mu=5
## n=10    0    1    2    3    4    5
## n=20    0    1    2    3    4    5
## 
## , , sd=1, rep=3
## 
##            mu=0     mu=1     mu=2     mu=3     mu=4     mu=5
## n=10  0.3012704 1.752093 1.451813 2.815746 3.860342 4.933741
## n=20 -0.1036849 1.066900 1.727783 2.844167 4.284351 4.876313
## 
## , , sd=2, rep=3
## 
##           mu=0      mu=1     mu=2     mu=3     mu=4     mu=5
## n=10 0.2150016 0.3657670 2.492447 3.937548 4.332272 5.816431
## n=20 0.4667103 0.8742835 1.616299 3.028465 3.845771 5.456615
## 
## , , sd=3, rep=3
## 
##            mu=0      mu=1     mu=2     mu=3     mu=4     mu=5
## n=10 -0.4774057 0.1070965 1.041310 2.135599 4.200848 4.426807
## n=20 -0.6590698 0.7566315 2.341365 4.161599 3.833062 4.154558
\end{verbatim}

We see, that an array with the right dimensions is created.

\hypertarget{monte-carlo-simulation-funcion}{%
\section{Monte Carlo Simulation
Funcion}\label{monte-carlo-simulation-funcion}}

\hypertarget{examples}{%
\section{Examples}\label{examples}}

\hypertarget{conclusion}{%
\section{Conclusion}\label{conclusion}}

The above section illustrates the power of our implemented model and
gives the fairly easy to use tool, that still allows for a variety of
different specifications in terms of used parameters, data generation
processes and summary functions. Researchers, who use Monte Carlo studys
on a regular basis, may save a lot of time using a tool like this in the
long run.

By nature, there may be cases, where our implementation doesnt satisfy
the needs of the user to the fullest, but for a wide variety of examples
we showed, that it worked well and served the goal that we aimed for.
Our functional programming approach allows for easy and flexible
adjustments in case the use of our functions should be expanded, f.e. if
a grid of more than 3 (or 4?) parameters is needed.

Theoretically, this work could be implemented as an R package to share
it with the R community. But since the \texttt{MonteCarlo()} function of
the \texttt{vigniette} package already provides a well working
alternative to our project besides some minor differences, there is
currently no need in doing that.

\hypertarget{references}{%
\section{References}\label{references}}

\hypertarget{contributions}{%
\section{Contributions}\label{contributions}}

\pagebreak
\renewcommand*{\mkbibnamefamily}[1]{\textbf{#1}}
\renewcommand*{\mkbibnamegiven}[1]{\textbf{#1}}
\renewcommand*{\mkbibnameprefix}[1]{\textbf{#1}}
\renewcommand*{\mkbibnamesuffix}[1]{\textbf{#1}}
\printbibliography[title=References]

\newpage
\textbf{Eidesstattliche Versicherung}

\bigskip

Ich versichere an Eides statt durch meine Unterschrift, dass ich die vorstehende Arbeit selbständig und ohne fremde Hilfe angefertigt und alle Stellen, die ich wörtlich oder annähernd wörtlich aus Veröffentlichungen entnommen habe, als solche kenntlich gemacht habe, mich auch keiner anderen als der angegebenen Literatur oder sonstiger Hilfsmittel bedient habe. Die Arbeit hat in dieser oder ähnlicher Form noch keiner anderen Prüfungsbehörde vorgelegen.

\vspace{1cm}
\rule{0pt}{2\baselineskip} %
\par\noindent\makebox[2.25in]{\indent Essen, den \hrulefill} \hfill\makebox[2.25in]{\hrulefill}%
\par\noindent\makebox[2.25in][l]{} \hfill\makebox[2.25in][c]{Alexander
Langnau, Öcal Kaptan, Sunyoung Ji}%


\end{document}
